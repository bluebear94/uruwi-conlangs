\documentclass{book}

\usepackage[shortsuper,hacm,ltfont]{common/uruwi}

\newcommand{\lname}{aaaaaaaaaaaaa}

\title{aaaaaaaaaaaaaaaaaaaaaa}
\author{uruwi}

\begin{document}

\pagecolor{Chartreuse1!25}

\begin{titlepage}
  \makeatletter
  \begin{center}
    {\color{Green3} \hprule \vspace{1.5ex} \\}
    %{\Huge \ltfont \textcolor{Plum}{LKe\bs{}TSxaRMoa SL LKeMa SL STfXe\bs{}RMyaKo}\\}
    {\Huge \sffamily \textcolor{SpringGreen4}{\@title} \\}
    {\large \textit{\lname}, the language of \textit{???} \\}
    {\color{Green3} \hprule \vspace{1.5ex} \\}
    % ----------------------------------------------------------------
    \vspace{1.5cm}
    {\Large\bfseries \@author}\\[5pt]
    %uruwi@protonmail.com\\[14pt]
    % ----------------------------------------------------------------
    \vspace{2cm}
    %{\Large\textlt{IVvoQMxeBPieLBxf TXxeKy}} \\
    \textnormal{een\^gs.-meibpelbe-kona} \\[5pt]
    \emph{A complete grammar}\\[2cm]
    %{in partial fulfilment for the award of the degree of} \\[2cm]
    %\tsc{\Large{{Doctor of Philosophy}}} \\[5pt]
    %{in some subject} \vspace{0.4cm} \\[2cm]
    % {By}\\[5pt] {\Large \sc {Me}}
    \vfill
    % ----------------------------------------------------------------
    %\includegraphics[width=0.19\textwidth]{example-image-a}\\[5pt]
    %{blah}\\[5pt]
    %{blahblah}\\[5pt]
    %{blahblah}\\
    \vfill
    {\@date}
  \end{center}
  \makeatother
\end{titlepage}

\pagecolor{Chartreuse1!15}

\begin{center}
    \textit{Dedicated to pecan.}
\end{center}

\begin{verbatim}
Branch: canon
Version: 0.1
Date: 2018-01-27 (29 lis vio)
\end{verbatim}

(C)opyright 2017 Uruwi. See README.md for details.

\tableofcontents

\section{Introduction}

\chapter{Phonology and orthography}

\section{Phoneme inventory and romanisation}

\lname uses the following phonemes:

\begin{table}[h]
  \caption{The consonants of \lname{}.}
  \centering
  \begin{tabular}{l|llllll}
      & Linguolabial & Alveolar & Palatal \\
      \hline
      Plosive & p b /t̼ d̼/ & t d /t d/ & k g /c ɟ/ \\
      Fricative & f /θ̼/ & þ /θ̠/ & \\
      (sibilant) & & s /s/ & \\
      (lateral) & ḟ /ɬ̼/ & ṡ ż /ɬ ɮ/ & \\
      (nareal) & & h /n̥͋/ & \\
      Approximant & & r /ɹ̥/ & j /j/ \\
      Nasal & m /n̼/ & & n /ɲ/ \\
  \end{tabular}
\end{table}
\begin{table}[h]
\centering
  \caption{The vowels of \lname.}
  \begin{tabular}{l|lll}
    & Front & Central & Back \\
    \hline
    High & i î /i iː/ & & u û /u uː/ \\
    Mid & & & o ô /ɔ oː/ \\
    Low & & a â /a aː/ & \\
  \end{tabular}
\end{table}

In addition, short vowels other than /i/ can be diphthongised with /i̯/ = /j/ to form, for instance, \ortho{aj} /ai̯/ or \ortho{ja} /i̯a/. However, at the start of a syllable, /j/ is treated as a consonant.

\section{Phonotactics}

Syllables are composed of:

\begin{itemize}
  \item An onset -- either:
  \begin{itemize}
    \item any consonant other than \ortho{m} /n̼/
    \item one of \ortho{sþ tþ tr þr sr hr þt ht rt þd rṡ rḟ þtr htr} /sθ̠ tθ̠ tɹ̥ θ̠ɹ̥ sɹ̥ n̥͋ɹ̥ θ̠t n̥͋t ɹ̥t ɹ̥ɬ θ̠d ɹ̥ɬ̼ θ̠tɹ̥ n̥͋tɹ̥/
    \item at the beginning of the word, nothing at all, but any medial syllables with an empty onset will receive an epethentic [ɹ̥] at that position.
  \end{itemize}
  \item A rime – one of:
  \begin{itemize}
    \item A long vowel or diphthong, and nothing else (diphthongs with onglides are not allowed if the onset is \ortho{j} /j/)
    \item A short vowel followed by one of \ortho{þ m r} /θ̠ n̼ ɹ̥/
  \end{itemize}
\end{itemize}

Double instances of a consonant between syllables are resolved as such:

\begin{itemize}
  \item /θ̠.θ̠/ → [θ̠.t]
  \item /θ̠.θ̠t/ → [θ̠.tʼ]
  \item /ɹ̥.ɹ̥/ → [ɹ̥.l̥]
\end{itemize}

These are not respelt.

\section{Allophony}

The following rules are applied:

\begin{alignat*}{3}
  % \alpha &\rightarrow \omega &\quad(\lambda \blacklozenge \rho) &\quad[\Gamma]
  \{\text{u}, \text{uː}\} &\rightarrow \{\text{y}, \text{ʉː}\} &\quad(C_1\{+ll\} \blacklozenge) \\
  \{\text{ɔ}, \text{oː}\} &\rightarrow \{\text{ɔ̜}, \text{ɵ̜u̯}\} &\quad(C_1\{+ll\} \blacklozenge) \\
  \text{ɹ̥} &\rightarrow \text{ɹ̼̊} &\quad(\blacklozenge C_1[+ll]) \\
  \{\text{t}, \text{d}\} &\rightarrow \{\text{t̼}, \text{d̼}\} &\quad(\text{n̼} \blacklozenge) \\
  \text{n̼} &\rightarrow \text{n} &\quad(\blacklozenge C_1[+av]) \\
  \text{θ̠} &\rightarrow \text{θ̠ə̆} &\quad(\blacklozenge C_1[+ll]) \\
  \text{θ̠} &\rightarrow \text{θ̠ʲ} &\quad(\blacklozenge \{\text{a}, \text{aː}, \text{ɔ}, \text{oː}\})
    &\quad\left[\text{frac}\left(\sqrt{\#\sigma + \#C}\right) < 0.5\right] \\
  \text{θ̠} &\rightarrow \text{ð̠} &\quad(\blacklozenge C_1[+v]) \\
  C_1[-v] &\rightarrow C_1[+a] &\quad(\blacklozenge V_1[+s]) \\
\end{alignat*}

\section{Stress}

Much like Ḋraħýl Rase, \lname{} has the concept of natural stress. That is, if syllables with short vowels are considered short and those with long vowels or diphthongs are long, then:

\begin{itemize}
  \item if the penultimate syllable is long, then it is stressed
  \item if the antepenultimate syllable is long, then it is stressed
  \item if the ultimate syllable is long, then it is stressed
  \item otherwise, the penultimate syllable is stressed
\end{itemize}

However, \lname{} is less free with deviations from this pattern. Notably, if the last three syllables are short, then the antepenultimate syllable can receive the stress instead. In the romanisation, this is marked with an acute accent.

\chapter{Syntax}

The basic word order of \lname{} is one of $\{N_1VN_2, N_1N_2V, VN_1N_2, N_1V, VN_2\}$. $N_1$ and $N_2$ are the ``subject'' and ``object'' of a verb, in either order.

Adjectives are placed farther from the verb than their antecedents. If an adjective $A$ modifies an $N$, then the onsets of $A$ and $N$ are switched. Adverbs occur at either the beginning or the end of the clause.

\section{Pivots}

When two clauses $\alpha$ and $\beta$ are joined by a clausal conjunction, some arguments may be omitted in the second clause.

\begin{itemize}
  \item If $\beta.N_1$ is omitted, then it defaults to $\alpha.N_2$ (inner pivot).
  \item If $\beta.N_2$ is omitted, then it defaults to $\alpha.N_1$ (outer pivot).
  \item If $\beta.V$ is omitted, then it defaults to $\alpha.V$ (verb pivot).
\end{itemize}

Noun phrases joined by nominal conjunctions work differently. The rules for those that occur after the verb are listed:

\begin{align}
  N_1^i A_1^i + N_2^j A_2^j &\rightarrow N_1 A_1 + N_2 A_2 \\
  N_1^i A_1^j + N_2^j A_2^i &\rightarrow (N_1 + N_2) (A_1 + A_2) \\
  N_1^i A^i + N_2 &\rightarrow N_1 A + N_2 A \\
  N_1 A^i + N_2^i &\rightarrow (N_1 + N_2) A \\
  N^i A_1^i +^j A_2^j &\rightarrow N A_1 + N A_2 \\
  N^i A_1^j +^j A_2^i &\rightarrow N (A_1 + A_2) \\
  N_1 + N_2^i A^i &\rightarrow N_1 A + N_2 \emptyset
\end{align}

The sequences are reversed before the verb.

For instance, using \ortho{ki} \emph{and}\footnote{There are two meanings that correspond to English's \emph{and}. The first is a bundle of both arguments present; the second is an object that has the properties of both arguments. Consider \emph{The dog and bird are a mammal and\,1 bipedal} versus \emph{The human is a mammal and\,2 bipedal}. \ortho{ki} uses the former interpretation for nouns and the latter for adjectives. \ortho{aþ} uses the latter interpretation for both nouns and adjectives.}, \ortho{napu} \emph{fish}, \ortho{pjiko} \emph{cat}, \ortho{karaha} \emph{fast} and \ortho{dombu} \emph{heavy}, we have the following (assuming that these NPs follow the verb):

\begin{itemize}
  \item \ortho{\hliii{kapu naraha} ki \hliv{djiko pombu}} \emph{the fast fish and the heavy cat}
  \item \ortho{\hliii{dapu} \hliv{paraha} ki \hliv{njiko} \hliii{kombu}} \emph{the (fast and heavy) (fish and cat)}
  \item \ortho{\hliii{kapu naraha} ki pjiko} \emph{the fast fish and the fast cat}
  \item \ortho{napu \hliii{paraha} ki \hliii{kjiko}} \emph{the fast (fish and cat)}
  \item \ortho{\hliii{kapu naraha} \hliv{di kombu}} \emph{the fast fish and the heavy fish}
  \item \ortho{\hliii{kapu} \hliv{daraha} \hliii{ni} \hliv{kombu}} \emph{the (fast and heavy) fish}
  \item \ortho{napu ki \hliii{kjiko paraha}} \emph{the (fast fish) and the cat}
\end{itemize}

\chapter{Nouns}

\section{Conceptualisation}

Nouns are declined for the following categories:

\begin{itemize}
  \item number-emergence
  \item similitude
  \item specificity
\end{itemize}

\subsection{Number-emergence}

Number and emergence (cf. \emph{The Avonian Language} 8.13.1 40) describes not only the quantity of an object but also any additional properties borne by the group.

\begin{itemize}
  \item \emph{Unmarked} is the default form of the noun.
  \item \emph{Reduced} is closest to English's plural form and confers no additional properties to a group of objects. Compared with the unmarked N-E, the reduced N-E is used the most often with human nouns, less often with other animates and rarely with inanimate nouns. With uncountable nouns, the unmarked form is always used.
  \item \emph{Emergent} describes a group of objects with properties extending beyond its components but also individual qualities.
  \item \emph{Coherent} describes an entity that cannot be meaningfully divided into its individual parts.
\end{itemize}

\subsection{Similitude}

Similitude (cf. \emph{The Avonian Language} 8.13.2 41) describes the differences among different objects of a group. This category is not marked in unmarked-NE nouns.

\begin{itemize}
  \item \emph{Identical} means that a similar group with $n$ elements as the one mentioned fall into $O(\log \log n)$ identities (to a margin of error) -- e.~g. a pile of candies or an orchard of apple trees.
  \item \emph{Similative} means that the members of the group in question are similar in name only -- e.~g. a forest with different species of plants.
  \item \emph{Related} means that at least one entity is the item in question and the others are related to it -- e.~g. spoons and other utensils.
\end{itemize}

\subsection{Specificity}

This refers to whether a noun phrase is unique in a given context and has two values: \emph{specific} and \emph{nonspecific}.

\section{Application}

In \lname{}, the default form of the noun is the \emph{unmarked similative specific} form.

% \ortho{napu} \emph{fish}

\subsection{The stem and the ending}

The ending of a noun is the rime of the last syllable, and the stem everything before that. For instance, the ending of \ortho{napu} \emph{fish} is \ortho{-u} and its stem \ortho{nap-}.

\subsection{Stem alternation}

If the noun is not stressed on the last syllable, then the consonant cluster after the stressed vowel is lenited as shown in table \ref{table:stemalt} in the weak form of the stem.

\begin{table}[ht]
  \caption{Lenitions of consonant clusters. \label{table:stemalt}}
  \centering
  \begin{tabular}{r|llll|l|llll}
    Onset \bs{} Coda & ∅ & þ & m & r & Onset \bs{} Coda & ∅ & þ & m & r \\
    \hline
    p & b & f & mb & rb & n & h & þn & n & n \\
    b & b & þ & ḟ & f & sþ & s & þþ & f & rþ \\
    f & ḟ & f & mb & rf & tþ & þ & þþ & mþ & rþ \\
    ḟ & ḟ & ḟ & mḟ & rḟ & tr & g & þr & mþr & rr \\
    t & d & þ & md & rd & þr & r & þr & mþ & rþ \\
    d & r & þ & md & r & sr & r & þr & md & rr \\
    þ & þ & þ & f & rd & hr & r & þr & mr & rr \\
    s & ṡ & þ & f & r & þt & þ & þt & mþ & rþ \\
    ṡ & ż & ṡ & mż & rż & ht & t & þt & mt & rt \\
    ż & r & ż & mż & r & rt & rd & þr & md & rd \\
    h & n & þn & n & r & þd & þ & þþ & mþ & rþ \\
    r & r & dr & r & r & rṡ & rż & þż & mż & rż \\
    k & g & þg & n & rg & rḟ & rḟ & þḟ & mḟ & rb \\
    g & g & þg & n & r & þtr & þr & þtr & mþr & rþr \\
    j & j & \invalid & \invalid & \invalid & htr & tr & þtr & mtr & rtr \\
  \end{tabular}
\end{table}

For instance, the weak form of the stem for \ortho{napu} is \ortho{nab-}. Similarly, the weak form of the stem for \ortho{dombu} is \ortho{doḟ-}.

\subsection{Vowel mutation}

An additional vowel mutation might be performed on the nucleus of the syllable before the stressed syllable ­-- see table \ref{table:vowelmutation}. Long vowels mutate similarly, as do diphthongs (which use the core vowel).

\begin{table}
  \caption{Vowel mutation. \label{table:vowelmutation}}
  \centering
  \begin{tabular}{r|llll}
    Pre-stressed \bs{} Stressed & a & i & o & u \\
    \hline
    a & i & u & i & o \\
    i & o & a & a & u \\
    o & u & i & a & a \\
    u & a & o & i & i \\
  \end{tabular}
\end{table}

\subsection{Declension tables}

In tables \ref{table:declensionfirst} to \ref{table:declensionlast}, S and W represent the strong and weak stems, respectively, and an asterisk denotes the presence of vowel mutation.

\begin{tablenf}
  \caption{Nouns that end with \ortho{-a}. \label{table:declensionfirst}}
  \centering
  \begin{tabular}{r|lll|lll}
    Specificity & \multicolumn{3}{c|}{Specific} & \multicolumn{3}{c}{Nonspecific} \\
    N-E \bs{} Sim. & Identical & Similar & Related & Identical & Similar & Related \\
    \hline
    Unmarked & \multicolumn{3}{c|}{S -a} & \multicolumn{3}{c}{S -ata} \\
    Reduced & S -a* & S -ar & S -aja & S -ata* & S -ara & S -ani \\
    Emergent & W -i & W -iþ & W -ija & W -ita & W -iþta & W -iji \\
    Coherent & W -i* & W -ir* & W -irja* & W -iþra* & W -iþta* & W -irji* \\
  \end{tabular}
\end{tablenf}

\begin{tablenf}
  \caption{Nouns that end with any other vowel \ortho{-V}.}
  \centering
  \begin{tabular}{r|lll|lll}
    Specificity & \multicolumn{3}{c|}{Specific} & \multicolumn{3}{c}{Nonspecific} \\
    N-E \bs{} Sim. & Identical & Similar & Related & Identical & Similar & Related \\
    \hline
    Unmarked & \multicolumn{3}{c|}{S -V} & \multicolumn{3}{c}{S -Vhta} \\
    Reduced & S -V* & S -Vr & S -Vja & S -Vta* & S -Vr & S -Vni \\
    Emergent & W -a & W -aj & W -aju & W -ata & W -aþa & W -aji \\
    Coherent & W -a* & W -ar* & W -aju* & W -aþra* & W -aþta* & W -arja* \\
  \end{tabular}
\end{tablenf}

\begin{tablenf}
  \caption{Nouns that end with \ortho{-aþ}.}
  \centering
  \begin{tabular}{r|lll|lll}
    Specificity & \multicolumn{3}{c|}{Specific} & \multicolumn{3}{c}{Nonspecific} \\
    N-E \bs{} Sim. & Identical & Similar & Related & Identical & Similar & Related \\
    \hline
    Unmarked & \multicolumn{3}{c|}{S -aþ} & \multicolumn{3}{c}{W -aþa} \\
    Reduced & S -aþ* & S -ar* & S -aþja* & W -ata* & W -aþra & W -ahi \\
    Emergent & S -a & S -aj & S -aja & W -ita & W -iþa & W -iji \\
    Coherent & S -iþ* & S -ar & S -ajaþ & W -iþraþ & W -iþta & W -irjaþ \\
  \end{tabular}
\end{tablenf}

\begin{tablenf}
  \caption{Nouns that end with \ortho{-iþ}.}
  \centering
  \begin{tabular}{r|lll|lll}
    Specificity & \multicolumn{3}{c|}{Specific} & \multicolumn{3}{c}{Nonspecific} \\
    N-E \bs{} Sim. & Identical & Similar & Related & Identical & Similar & Related \\
    \hline
    Unmarked & \multicolumn{3}{c|}{S -iþ} & \multicolumn{3}{c}{W -iþa} \\
    Reduced & S -iþ* & S -ir* & S -iþja* & W -ita* & W -iþra & W -ihu \\
    Emergent & S -i & S -ij & S -ija & W -ata & W -iþa & W -iji \\
    Coherent & S -oþ* & S -ir & S -ijaþ & W -aþraþ & W -aþta & W -arjaþ \\
  \end{tabular}
\end{tablenf}

\begin{tablenf}
  \caption{Nouns that end with \ortho{-oþ}.}
  \centering
  \begin{tabular}{r|lll|lll}
    Specificity & \multicolumn{3}{c|}{Specific} & \multicolumn{3}{c}{Nonspecific} \\
    N-E \bs{} Sim. & Identical & Similar & Related & Identical & Similar & Related \\
    \hline
    Unmarked & \multicolumn{3}{c|}{S -oþ} & \multicolumn{3}{c}{W -oþa} \\
    Reduced & S -oþ* & S -or* & S -oja* & W -uta* & W -oþra & W -ohu \\
    Emergent & S -o & S -oj & S -oja & W -ata & W -uþa & W -uju \\
    Coherent & S -iþ* & S -or & S -ojaþ & W -aþraþ & W -uþta & W -ujiþ \\
  \end{tabular}
\end{tablenf}

\begin{tablenf}
  \caption{Nouns that end with \ortho{-uþ}.}
  \centering
  \begin{tabular}{r|lll|lll}
    Specificity & \multicolumn{3}{c|}{Specific} & \multicolumn{3}{c}{Nonspecific} \\
    N-E \bs{} Sim. & Identical & Similar & Related & Identical & Similar & Related \\
    \hline
    Unmarked & \multicolumn{3}{c|}{S -uþ} & \multicolumn{3}{c}{W -uþa} \\
    Reduced & S -uþ* & S -ur* & S -uja* & W -uta* & W -uþra & W -uha \\
    Emergent & S -i & S -ij & S -uja & W -ata & W -aþa & W -aju \\
    Coherent & S -iþ* & S -ar & S -ajaþ & W -aþraþ & W -aþta & W -ajiþ \\
  \end{tabular}
\end{tablenf}

\begin{tablenf}
  \caption{Nouns that end with \ortho{-Vm}. \label{table:declensionlast}}
  \centering
  \begin{tabular}{r|lll|lll}
    Specificity & \multicolumn{3}{c|}{Specific} & \multicolumn{3}{c}{Nonspecific} \\
    N-E \bs{} Sim. & Identical & Similar & Related & Identical & Similar & Related \\
    \hline
    Unmarked & \multicolumn{3}{c|}{S -Vm} & \multicolumn{3}{c}{W -Vm} \\
    Reduced & S -Vha & W -Vha* & W -Vþ & S -ar & S -ihV* & W -ihV \\
    Emergent & S -Vþ & W -Vþ* & S -Vha* & W -ihV* & S -ar* & W -ar \\
    Coherent & S -Vja & W -Vha & S -Vþ* & W -ar* & S -ihV & W -ajiþ \\
  \end{tabular}
\end{tablenf}

\begin{tablenf}
  \caption{Nouns that end with \ortho{-Vr}.}
  \centering
  \begin{tabular}{r|lll|lll}
    Specificity & \multicolumn{3}{c|}{Specific} & \multicolumn{3}{c}{Nonspecific} \\
    N-E \bs{} Sim. & Identical & Similar & Related & Identical & Similar & Related \\
    \hline
    Unmarked & \multicolumn{3}{c|}{S -Vr} & \multicolumn{3}{c}{W -Vr} \\
    Reduced & S -Vra & W -Vra* & W -Vþ & S -an & S -inV* & W -imV \\
    Emergent & S -Vþ & W -Vþ* & S -Vra* & W -inV* & S -an* & W -am \\
    Coherent & S -Vja & W -Vra & S -Vþ* & W -an* & S -inV & W -ariþ \\
  \end{tabular}
\end{tablenf}

\begin{tablenf}
  \caption{Nouns that end with \ortho{-â}}.
  \centering
  \begin{tabular}{r|lll|lll}
    Specificity & \multicolumn{3}{c|}{Specific} & \multicolumn{3}{c}{Nonspecific} \\
    N-E \bs{} Sim. & Identical & Similar & Related & Identical & Similar & Related \\
    \hline
    Unmarked & \multicolumn{3}{c|}{S -â} & \multicolumn{3}{c}{S -âta} \\
    Reduced & S -âri & S -âri & S -ara & S -âra & S -âra & S -ara \\
    Emergent & S -âri & S -aj & S -ara & S -âna & S -âna & S -ajra \\
    Coherent & S -aja & S -aþ & S -ara & S -âna & S -âna & S -aþ \\
  \end{tabular}
\end{tablenf}

\begin{tablenf}
  \caption{Nouns that end with \ortho{-î}}.
  \centering
  \begin{tabular}{r|lll|lll}
    Specificity & \multicolumn{3}{c|}{Specific} & \multicolumn{3}{c}{Nonspecific} \\
    N-E \bs{} Sim. & Identical & Similar & Related & Identical & Similar & Related \\
    \hline
    Unmarked & \multicolumn{3}{c|}{S -î} & \multicolumn{3}{c}{S -îta} \\
    Reduced & S -îri & S -îri & S -ari & S -îra & S -îra & S -ari \\
    Emergent & S -îþri & S -iþ & S -ira & S -îna & S -îna & S -îra \\
    Coherent & S -ija & S -iþ & S -ira & S -îna & S -îna & S -iþ \\
  \end{tabular}
\end{tablenf}

\begin{tablenf}
  \caption{Nouns that end with \ortho{-ô}}.
  \centering
  \begin{tabular}{r|lll|lll}
    Specificity & \multicolumn{3}{c|}{Specific} & \multicolumn{3}{c}{Nonspecific} \\
    N-E \bs{} Sim. & Identical & Similar & Related & Identical & Similar & Related \\
    \hline
    Unmarked & \multicolumn{3}{c|}{S -ô} & \multicolumn{3}{c}{S -ôpa} \\
    Reduced & S -ôri & S -ôri & S -om & S -ôra & S -ôra & S -ora \\
    Emergent & S -ôri & S -oj & S -om & S -ôna & S -ôna & S -or \\
    Coherent & S -oj & S -oþ & S -om & S -ôna & S -ôna & S -aþ \\
  \end{tabular}
\end{tablenf}

\begin{tablenf}
  \caption{Nouns that end with \ortho{-û}. \label{table:declensionlast}}
  \centering
  \begin{tabular}{r|lll|lll}
    Specificity & \multicolumn{3}{c|}{Specific} & \multicolumn{3}{c}{Nonspecific} \\
    N-E \bs{} Sim. & Identical & Similar & Related & Identical & Similar & Related \\
    \hline
    Unmarked & \multicolumn{3}{c|}{S -û} & \multicolumn{3}{c}{S -ûpa} \\
    Reduced & S -ûri & S -ûri & S -um & S -ûfa & S -ûfa & S -ura \\
    Emergent & S -ûrḟi & S -uj & S -um & S -ûna & S -ûna & S -ur \\
    Coherent & S -uþ & S -uþ & S -um & S -umba & S -umba & S -aþ \\
  \end{tabular}
\end{tablenf}

\subsection{Examples}

\begin{tablenf}
  \caption{Declensions for \ortho{napu} \emph{fish}. \label{table:declensionex1}}
  \centering
  \begin{tabular}{r|lll|lll}
    Specificity & \multicolumn{3}{c|}{Specific} & \multicolumn{3}{c}{Nonspecific} \\
    N-E \bs{} Sim. & Identical & Similar & Related & Identical & Similar & Related \\
    \hline
    Unmarked & \multicolumn{3}{c|}{napu} & \multicolumn{3}{c}{napuhta} \\
    Reduced & napa & napur & napuja & noputa & napur & napuni \\
    Emergent & naba & nabaj & nabaju & nabata & nabaþa & nabaji \\
    Coherent & nabi & nabir & nibaju & nibaþra & nibaþta &nibarja \\
  \end{tabular}
\end{tablenf}

\begin{tablenf}
  \caption{Declensions for \ortho{hrisaþ} \emph{rock}. \label{table:declensionex1}}
  \centering
  \begin{tabular}{r|lll|lll}
    Specificity & \multicolumn{3}{c|}{Specific} & \multicolumn{3}{c}{Nonspecific} \\
    N-E \bs{} Sim. & Identical & Similar & Related & Identical & Similar & Related \\
    \hline
    Unmarked & \multicolumn{3}{c|}{hrisaþ} & \multicolumn{3}{c}{hriṡaþa} \\
    Reduced & hrisuþ & hrisur & hrosaþja & hroṡata & hriṡaþra & hriṡahi \\
    Emergent & hrisa & hrisaj & hrisaja & hriṡita & hriṡiþa & hriṡiji \\
    Coherent & hrisaþ & hrisar & hrisajaþ & hriṡiþraþ & hriṡiþta & hriṡirjaþ \\
  \end{tabular}
\end{tablenf}

\end{document}