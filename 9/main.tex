\documentclass{book}

\usepackage[shortsuper,hacm,ltfont]{common/uruwi}

\newcommand{\lname}{aaaaaaaaaaaaa}

\title{aaaaaaaaaaaaaaaaaaaaaa}
\author{uruwi}

\begin{document}

\pagecolor{Chartreuse1!25}

\begin{titlepage}
  \makeatletter
  \begin{center}
    {\color{Green3} \hprule \vspace{1.5ex} \\}
    %{\Huge \ltfont \textcolor{Plum}{LKe\bs{}TSxaRMoa SL LKeMa SL STfXe\bs{}RMyaKo}\\}
    {\Huge \sffamily \textcolor{SpringGreen4}{\@title} \\}
    {\large \textit{\lname}, the language of \textit{???} \\}
    {\color{Green3} \hprule \vspace{1.5ex} \\}
    % ----------------------------------------------------------------
    \vspace{1.5cm}
    {\Large\bfseries \@author}\\[5pt]
    %uruwi@protonmail.com\\[14pt]
    % ----------------------------------------------------------------
    \vspace{2cm}
    %{\Large\textlt{IVvoQMxeBPieLBxf TXxeKy}} \\
    \textnormal{een\^gs.-meibpelbe-kona} \\[5pt]
    \emph{A complete grammar}\\[2cm]
    %{in partial fulfilment for the award of the degree of} \\[2cm]
    %\tsc{\Large{{Doctor of Philosophy}}} \\[5pt]
    %{in some subject} \vspace{0.4cm} \\[2cm]
    % {By}\\[5pt] {\Large \sc {Me}}
    \vfill
    % ----------------------------------------------------------------
    %\includegraphics[width=0.19\textwidth]{example-image-a}\\[5pt]
    %{blah}\\[5pt]
    %{blahblah}\\[5pt]
    %{blahblah}\\
    \vfill
    {\@date}
  \end{center}
  \makeatother
\end{titlepage}

\pagecolor{Chartreuse1!15}

\begin{center}
    \textit{Dedicated to Gufferdk.}
\end{center}

\begin{verbatim}
Branch: canon
Version: 0.1
Date: 2018-01-27 (29 lis vio)
\end{verbatim}

(C)opyright 2017 Uruwi. See README.md for details.

\tableofcontents

\section{Introduction}

\chapter{Phonology and orthography}

\section{Phoneme inventory and romanisation}

\lname uses the following phonemes:

\begin{table}[h]
  \caption{The consonants of \lname{}.}
  \centering
  \begin{tabular}{l|llllll}
      & Linguolabial & Alveolar & Palatal \\
      \hline
      Plosive & p b /t̼ d̼/ & t d /t d/ & k g /c ɟ/ \\
      Fricative & f /θ̼/ & þ /θ̠/ & \\
      (sibilant) & & s /s/ & \\
      (lateral) & ḟ /ɬ̼/ & ṡ ż /ɬ ɮ/ & \\
      (nareal) & & h /n̥͋/ & \\
      Approximant & & r /ɹ̥/ & j /j/ \\
      Nasal & m /n̼/ & & n /ɲ/ \\
  \end{tabular}
\end{table}
\begin{table}[h]
\centering
  \caption{The vowels of \lname.}
  \begin{tabular}{l|lll}
    & Front & Central & Back \\
    \hline
    High & i î /i iː/ & & u û /u uː/ \\
    Mid & & & o ô /ɔ oː/ \\
    Low & & a â /a aː/ & \\
  \end{tabular}
\end{table}

In addition, short vowels other than /i/ can be diphthongised with /i̯/ = /j/ to form, for instance, \ortho{aj} /ai̯/ or \ortho{ja} /i̯a/. However, at the start of a syllable, /j/ is treated as a consonant.

\section{Phonotactics}

Syllables are composed of:

\begin{itemize}
  \item An onset: any consonant other than \ortho{m} /n̼/
  \item A rime – one of:
  \begin{itemize}
    \item A long vowel or diphthong, and nothing else (diphthongs with onglides are not allowed if the onset is \ortho{j} /j/)
    \item A short vowel followed by one of \ortho{þ m r} /θ̠ n̼ ɹ̥/
  \end{itemize}
\end{itemize}

Double instances of a consonant between syllables are resolved as such:

\begin{itemize}
  \item /θ̠.θ̠/ → [θ̠.t]
  \item /n̼.n̼/ → [n̼.d̼]
  \item /ɹ̥.ɹ̥/ → [ɹ̥.l̥]
\end{itemize}

These are not respelt.

\section{Allophony}

The following rules are applied:

\begin{alignat*}{3}
  % \alpha &\rightarrow \omega &\quad(\lambda \blacklozenge \rho) &\quad[\Gamma]
  \{\text{u}, \text{uː}\} &\rightarrow \{\text{y}, \text{ʉː}\} &\quad(C_1\{+ll\} \blacklozenge) \\
  \{\text{ɔ}, \text{oː}\} &\rightarrow \{\text{ɔ̜}, \text{ɵ̜u̯}\} &\quad(C_1\{+ll\} \blacklozenge) \\
  \text{ɹ̥} &\rightarrow \text{ɹ̼̊} &\quad(\blacklozenge C_1\{+ll\}) \\
  \{\text{t}, \text{d}\} &\rightarrow \{\text{t̼}, \text{d̼}\} &\quad(\text{n̼} \blacklozenge) \\
  \text{n̼} &\rightarrow \text{n} &\quad(\blacklozenge C_1\{+av\}) \\
  \text{θ̠} &\rightarrow \text{θ̠ə̆} &\quad(\blacklozenge C_1\{+ll\}) \\
  \text{θ̠} &\rightarrow \text{θ̠ʲ} &\quad(\blacklozenge \{\text{a}, \text{aː}, \text{ɔ}, \text{oː}\})
    &\quad\left[\text{frac}\left(\sqrt{\#\sigma + \#C}\right) < 0.5\right] \\
\end{alignat*}

\end{document}