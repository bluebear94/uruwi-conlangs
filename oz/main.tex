\documentclass{book}

\usepackage[shortsuper,hacm,dhr]{common/uruwi}

\newcommand{\lname}{aaaaaaaaaaA}

\title{Uruwi's personal style guide for documents}
\author{uruwi}

\begin{document}

\pagecolor{Goldenrod!25}

\begin{titlepage}
    \makeatletter
    \begin{center}
        {\color{RedOrange} \hprule \vspace{1.5ex} \\}
        {\Huge \textcolor{BrickRed}{\@title}\\}
        %{\large \sffamily \textcolor{Purple}{\@title} \\}
        %{\large \textit{\lname}, the language of \textit{Rymako} \\}
        {\color{RedOrange} \hprule \vspace{1.5ex} \\}
        % ----------------------------------------------------------------
        \vspace{1.5cm}
        {\Large\bfseries \@author}\\[5pt]
        %uruwi@protonmail.com\\[14pt]
        % ----------------------------------------------------------------
        \vspace{2cm}
        %\textkardinal{aaaaaaaaaaaaaaaaa} \\
        %{aaaaaaaaaaaaaaaaa} \\[5pt]
        %\emph{A complete grammar}\\[2cm]
        %{in partial fulfilment for the award of the degree of} \\[2cm]
        %\tsc{\Large{{Doctor of Philosophy}}} \\[5pt]
        %{in some subject} \vspace{0.4cm} \\[2cm]
        % {By}\\[5pt] {\Large \sc {Me}}
        \vfill
        % ----------------------------------------------------------------
        %\includegraphics[width=0.19\textwidth]{example-image-a}\\[5pt]
        %{blah}\\[5pt]
        %{blahblah}\\[5pt]
        %{blahblah}\\
        \vfill
        {\@date}
    \end{center}
    \makeatother
\end{titlepage}

\pagecolor{Goldenrod!15}

\begin{verbatim}
Branch: canon
Version: 0.1
Date: 2017-10-07
\end{verbatim}

(C)opyright 2017 Uruwi. See README.md for details.

\tableofcontents

\chapter{Overview of build process}

To generate grammars and other documents, \XeLaTeX{} is used. Documents depend on the \texttt{common/uruwi.sty} package, which imports dependent packages such as \texttt{xcolor} and \texttt{tabu}, as well as defining in-house macros such as \texttt{\bs{}hli} and \texttt{\bs{}ortho}.

The build process is automated using \texttt{make}, which, in addition to invoking \XeLaTeX{} to build the document, generates \texttt{dict.tex} files from dictionary files.

\section{Lexicon management}

Lexicons are stored in \texttt{.dict} files, which are plain text files with some formatting info. A typical file will have entries like this:

\begin{lstlisting}
# ramek
: vn?
break, shatter, tear, destroy
n = what was broken was in the way; non-n = what did the breaking sought out things to break

# kekoro
: n
most, majority

# malka
: n
quiet, calm, sound

# rajnek
: v
sleep

# ranu
: n
fox

# kretanek
: v
run

# mepek
: v
learn, teach (about)
learn <A> → <A>-\textsf{kejm%*á*) mepek}
\end{lstlisting}

Evidently, each entry is delimited by one or more blank lines. A line starting with an octothorpe gives the entry in the target language. A line beginning with a colon defines the part of speech.

Other lines provide a definition. Some entries require multiple lines; in that case, the subsequent lines will act as usage notes or examples.

The dictionary file is converted to a \LaTeX file with the \texttt{dict-to-tex.pl6} script. This script also takes a JSON file that specifies styling information and the lexicographic ordering. For instance, the dictionary file for Lek-Tsaro uses this \texttt{options.json} file\footnote{I'm showing the Lek-Tsaro options file instead of the one used by Ḋraħýl Rase because the latter uses Unicode characters, which don't display quite properly with \texttt{listings.sty}.}:

\lstinputlisting[language=Octave]{7/dict/options.json}

\section{Historical tools}

Historically:

\begin{itemize}
  \item Google Docs was used for prototyping language grammars. However, the grammar of Lek-Tsaro was not prototyped in that manner.
  \item Google Sheets was used for managing lexicons. This was phased out over concerns of using propietary software.
\end{itemize}

\section{Document styling}

\subsection{Typefaces}

The \texttt{mathspec} package is used for custom fonts in both text- and mathmode. Gentium is used for the normal font, and \textsf{VL PGothic} for the sans-serif font. The monospace font is not set.

When hacm text is needed, the \texttt{uruwi.sty} package is loaded with the \texttt{hacm} option, which sets \texttt{\bs{}kardinal} to the \textkardinal{kardinal} font, modified to include the backslash character (used by Lek-Tsaro). Frequently, superscripts in text are also needed, so \texttt{uruwi.sty} is also loaded with the \texttt{shortsuper} option, which redefines \verb|\^| to the longer \texttt{\bs{textsuperscript}}.

If the \texttt{dhr} option is set, then \texttt{uruwi.sty} will also set \texttt{\bs{dhrfont}} to the \textdhr{mIny/meko} (Mîny / Meko) font, which supports \emph{Nasél Tēkel Piva}. A guide to using this font can be found in table \ref{table:ntpmm}.

\newcommand{\nrpair}[2]{#1 & #2 & #1}
\begin{table}[h]
  \caption{Guide to using the \textdhr{mIny/meko} font. \label{table:ntpmm}}
  \centering
  \begin{tabular}{|>{\dhrfont}l|l|>{\ttfamily}l||>{\dhrfont}l|l|>{\ttfamily}l||>{\dhrfont}l|l|>{\ttfamily}l|}
    \hline
    \textnormal{NTP} & Rom & \textnormal{Seq} &
    \textnormal{NTP} & Rom & \textnormal{Seq} &
    \textnormal{NTP} & Rom & \textnormal{Seq} \\
    \hline
    \nrpair{p}{p} &
    \nrpair{t}{t} &
    \nrpair{k}{k} \\
    \nrpair{s}{s} &
    \nrpair{f}{f} &
    \nrpair{n}{n} \\
    \nrpair{m}{m} &
    \nrpair{x}{ḣ} &
    \nrpair{H}{ħ} \\
    \nrpair{h}{h} &
    \nrpair{r}{r} &
    \nrpair{S}{ṡ} \\
    \nrpair{l}{l} &
    \nrpair{v}{v} &
    \nrpair{g}{g} \\
    \nrpair{N}{ṅ} &
    \nrpair{d}{d} &
    \nrpair{b}{b} \\
    \nrpair{Z}{ż} &
    \nrpair{z}{z} &
    \nrpair{G}{ġ} \\
    \nrpair{D}{ḋ} &
    \nrpair{T}{ṫ} &
    & & \\
    \hline
    \nrpair{A}{â} &
    \nrpair{E}{ê} &
    \nrpair{I}{î} \\
    \nrpair{O}{ô} &
    \nrpair{U}{û} &
    \nrpair{Y}{ŷ} \\
    \hline
    \nrpair{ta}{ta} &
    \nrpair{ra}{ra} &
    \nrpair{pa}{pa} \\
    \nrpair{te}{te} &
    \nrpair{re}{re} &
    \nrpair{pe}{pe} \\
    \nrpair{ti}{ti} &
    \nrpair{ri}{ri} &
    \nrpair{pi}{pi} \\
    \nrpair{to}{to} &
    \nrpair{ro}{ro} &
    \nrpair{po}{po} \\
    \nrpair{tu}{tu} &
    \nrpair{ru}{ru} &
    \nrpair{pu}{pu} \\
    \nrpair{ty}{ty} &
    \nrpair{ry}{ry} &
    \nrpair{py}{py} \\
    \nrpair{fa}{fa} &
    \nrpair{fe}{fe} &
    \nrpair{fi}{fi} \\
    \nrpair{fo}{fo} &
    \nrpair{fu}{fu} &
    \nrpair{fy}{fy} \\
    \hline
    \nrpair{Aj}{aj} &
    \nrpair{Ej}{ej} &
    & & \\
    \nrpair{Oj}{oj} &
    \nrpair{Uj}{uj} &
    \nrpair{Yj}{yj} \\
    \nrpair{jA}{ja} &
    \nrpair{jE}{je} &
    & & \\
    \nrpair{jO}{jo} &
    \nrpair{jU}{ju} &
    \nrpair{jY}{jy} \\
    \nrpair{Aw}{aw} &
    \nrpair{Ew}{ew} &
    \nrpair{Iw}{iw} \\
    \nrpair{Ow}{ow} &
    & & &
    \nrpair{Yw}{yw} \\
    \nrpair{wA}{wa} &
    \nrpair{wE}{we} &
    \nrpair{wI}{wi} \\
    \nrpair{wO}{wo} &
    & & &
    \nrpair{wY}{wy} \\
    \nrpair{AW}{aẏ} &
    \nrpair{EW}{eẏ} &
    \nrpair{IW}{iẏ} \\
    \nrpair{OW}{oẏ} &
    \nrpair{UW}{uẏ} &
    & & \\
    \nrpair{WA}{ẏa} &
    \nrpair{WE}{ẏe} &
    \nrpair{WI}{ẏi} \\
    \nrpair{WO}{ẏo} &
    \nrpair{WU}{ẏu} &
    & & \\
    \hline
    \nrpair{0}{0} &
    \nrpair{1}{1} &
    \nrpair{2}{2} \\
    \nrpair{3}{3} &
    \nrpair{4}{4} &
    \nrpair{5}{5} \\
    \nrpair{6}{6} &
    \nrpair{7}{7} &
    \nrpair{8}{8} \\
    \nrpair{9}{9} &
    \nrpair{:}{X} &
    \nrpair{;}{E} \\
    \hline
    \nrpair{.}{.} &
    \nrpair{,}{,} &
    \nrpair{?}{?} \\
    / & / & /\footnote{No space!} &
    \nrpair{\tl{}}{kêl} &
    \nrpair{[]}{``''} \\
    \hline
  \end{tabular}
\end{table}

\subsection{Pages}

The first page in a document is the title page. For a conlang, this consists of:

\begin{itemize}
  \item A horizontal rule.
  \item A phrase akin to ``(language), the language of (place)'' in the target language. If said language has its own script, then this should be written in the script, with a transliteration below it in a smaller, sans-serif font.
  \item The translation of the above, in italics.
  \item Another horizontal rule identical to one above.
  \item The name of the author.
  \item ``A complete grammar'', first in the native script (if present), then transliterated, then in English.
  \item The date, at the very bottom.
\end{itemize}

All of these entries are centred.

For documents other than language grammars, simply include what is relevant to the document.

The title page should be coloured at 25\% of one of the predefined colours in \texttt{xcolor}.

For more information, consult the title pages of other documents.

The second page contains an optional dedication, followed by metadata. The metadata is set in a monospace font with the following fields:

\begin{itemize}
  \item \texttt{Branch:} This is \texttt{canon} for the main branch, and different for experimental branches of the grammar.
  \item \texttt{Version:} A version, updated occasionally.
  \item \texttt{Date:} The date when this version was adopted.
\end{itemize}

After the metadata is the copyright information.

The second and subsequent pages should be coloured at 15\% of the same shade of colour as the title page (so if the title page is at \texttt{Thistle!25}, the body pages should be at \texttt{Thistle!15}).

\subsection{Colours}

The following guidelines are used for selecting page colours:

\begin{itemize}
  \item A daughter language should have the same or similar page colour as its parent.
  \item On the other hand, a language from a completely different family should have a visibly distinct page colour.
  \item If possible, select a colour that matches the character of the language the document covers.
  \item Avoid overly dark or light colours.
  \item Avoid grey or other drab colours.
  \item Favour cool colours (but don't actively avoid warm ones).
\end{itemize}

Any coloured content in the title page should blend with the page colour.

However, the chapter and section styles don't need to be changed; in fact, none of the grammars change their styles.

\subsection{Semantic styles}

Strings in the target language inside English text should be wrapped inside an \texttt{\bs{ortho}} (or \texttt{\bs{hortho}} for hacm text), \ortho{ħanaħâle-tuẏrí}.

\subsection{Example sentences}

Example sentences include the (transliterated) sentence, the gloss and the idiomatic translation. The target-language sentence and the gloss are not aligned; rather, each word is coloured in both the first and second lines. The words of the English translation are coloured to roughly match the target-language sentence: \\
~\\
\textkardinal{\hli{tsiltanke} \hlii{me} \hliii{sartama} \hliv{sil} \hlv{s\^ha.eke.}} \\
\hli{want-\tsc{near}-\tsc{neg}} \hlii{\tsc{pr.near.sg}} \hliii{ring=\tsc{gen}} \hliv{\tsc{pos}} \hlv{magician-\tsc{null}} \\
\hlii{I} \hli{don't want} \hliii{the rings} \hliv{of} \hlv{any magician.} \\

Underscores are preferred over periods in morphemes that take multiple words to describe:
~\\
\textkardinal{\hli{maki\^oha} \hlii{.u\^o-s\^hin} \hliii{txoro,} \hliv{doran} \hlv{jace} \hlvi{nyara} \hlvii{ra} \hlviii{net.}} \\
\hli{eat-\tsc{generic}-\tsc{q}} \hlii{\tsc{pr.generic}-how\_many} \hliii{flower,} \hliv{\tsc{cmp}-\tsc{near}} \hlv{fish} \hlvi{cat} \hlvii{\tsc{pr.anaph\_sub.int}} \hlviii{$>$} \\
\hlviii{More} \hlv{fish} \hli{eat} \hliii{flowers} \hliv{than} \hlvi{cats.} \\

\end{document}