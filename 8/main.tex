\documentclass{book}

\usepackage[shortsuper,hacm]{common/uruwi}

\newcommand{\lname}{aaaaaaaaaaA}

\title{???}
\author{uruwi}

\begin{document}

\pagecolor{GreenYellow!25}

\begin{titlepage}
    \makeatletter
    \begin{center}
        {\color{Orchid} \hprule \vspace{1.5ex} \\}
        {\Huge \kardinal \textcolor{Plum}{\@title}\\}
        %{\large \kardinal \textcolor{Purple}{\@title} \\}
        {\large \textit{\lname}, the language of \textit{somewhere} \\}
        {\color{Orchid} \hprule \vspace{1.5ex} \\}
        % ----------------------------------------------------------------
        \vspace{1.5cm}
        {\Large\bfseries \@author}\\[5pt]
        %uruwi@protonmail.com\\[14pt]
        % ----------------------------------------------------------------
        \vspace{2cm}
        \textkardinal{aaaaaaaaaaaaaaaaa} \\
        {aaaaaaaaaaaaaaaaa} \\[5pt]
        \emph{A complete grammar}\\[2cm]
        %{in partial fulfilment for the award of the degree of} \\[2cm]
        %\tsc{\Large{{Doctor of Philosophy}}} \\[5pt]
        %{in some subject} \vspace{0.4cm} \\[2cm]
        % {By}\\[5pt] {\Large \sc {Me}}
        \vfill
        % ----------------------------------------------------------------
        %\includegraphics[width=0.19\textwidth]{example-image-a}\\[5pt]
        %{blah}\\[5pt]
        %{blahblah}\\[5pt]
        %{blahblah}\\
        \vfill
        {\@date}
    \end{center}
    \makeatother
\end{titlepage}

\pagecolor{GreenYellow!15}

\begin{center}
    \textit{Dedicated to Mareck.}
\end{center}

\begin{verbatim}
Branch: canon
Version: 0.1
Date: 2017-11-12 (28 ruj fav)
\end{verbatim}

(C)opyright 2017 Uruwi. See README.md for details.

\tableofcontents

\section{Introduction}

\chapter{Phonology and orthography}

\section{Phoneme inventory}

\synopsis{Consonants are in free variation with vowels.}

In \lname{}, each consonant is interchangeable with a corresponding vowel. Consonants may also have an ingressive pronuniation.

\begin{table}[h]
  \caption{Phonemes of \lname.}
  \centering
  \begin{tabular}{r>{\kardinal}llll}
    & & \multicolumn{2}{c}{Consonant} & \\
    \# & \textnormal{Hacm} & (Egre) & (Ingr) & Vowel \\
    \hline
    0 & t & tʼ & ǃ & e˥ \\
    1 & n\^g & ŋ & ɠ & ã \\
    2 & b & p & p & y \\
    3 & m & m & ɓ & ũ \\
    4 & s & s & s & ɨ \\
    5 & y & j & j & i \\
    6 & x & tɬʼ & ǁ & ʌ˥ \\
    7 & w & w & w & u \\
    8 & g & k & k & o \\
    9 & l & l & l & ɯ \\
    10 & k & kʼ & ǂ & o˥ \\
    11 & n & n & n & e˥ \\
    12 & c & r & ɹ & ḛ \\
    13 & d & t & t & e \\
    14 & h & ħ & h & a \\
    15 & j & ɬ & ɬ & ʌ \\
   \end{tabular}
\end{table}

When pronounced ingressively, the tones of vowels are inverted. That is, [ʌ˥↑] becomes [ʌ˩↓].

\section{Airflow}

\synopsis{Change of airflow direction has a morphosyntactic basis.}

There are two types of airflow: \emph{ingressive} and \emph{egressive}. The direction of airflow is reversed:

\begin{itemize}
  \item at the beginning of a modifier
  \item at certain affixes
  \item in the middle of certain roots
\end{itemize}

On a proper noun, as well as on encountering a nasal vowel, the direction is switched to egressive and remains so until it is changed by one of the above methods.

In hacm, switching the direction of airflow is marked by \hortho{/} (to ingressive) and \hortho{\bs} (to egressive). In dictionaries, a switch in airflow direction (without regard to the final state) is marked using \hortho{>}.

\section{Phonotactics}

The only phonotactic restriction is that two identical instances of a phoneme may not occur consecutively. If this rule is violated by affixation, then the violation is resolved by:

\begin{itemize}
  \item replacing the earlier instance with an instance of its predecessor (e.~g. /w/ (7) $\rightarrow$ /tɬʼ/ (6), wrapping when necessary), and
  \item replacing the later instance with an instance of its successor (e.~g. /w/ (7) $\rightarrow$ /k/ (8), wrapping when necessary).
\end{itemize}

\section{Allophony}

The following changes are made:

\begin{alignat*}{2}
  % \alpha &\rightarrow \omega &\quad(\lambda \blacklozenge \rho) &\quad[\Gamma]
  \text{lm} &\rightarrow \text{p} &\quad \\
  \text{nl} &\rightarrow \text{r} &\quad \\
  \text{ɬ} &\rightarrow \text{tʼ} &\quad(\blacklozenge \lnot \{\square, \text{tʼ}, \text{kʼ}\})
\end{alignat*}

(Here, the symbols for the egressive versions of the consonants are used, but these rules apply during ingressive airflow as well.)

Thus, for instance, /ħswlmŋ/ would be resolved to [ħswpŋ], which could, for instance, be pronounced [asupã].

\end{document}