\documentclass{book}

\usepackage[shortsuper,hacm,ltfont]{common/uruwi}

\newcommand{\lname}{aaaaaaaaaaaaa}

\title{aaaaaaaaaaaaaaaaaaaaaa}
\author{uruwi}

\begin{document}

\pagecolor{LightSkyBlue2!25}

\begin{titlepage}
  \makeatletter
  \begin{center}
    {\color{DeepSkyBlue3} \hprule \vspace{1.5ex} \\}
    %{\Huge \ltfont \textcolor{Plum}{LKe\bs{}TSxaRMoa SL LKeMa SL STfXe\bs{}RMyaKo}\\}
    {\Huge \sffamily \textcolor{SteelBlue3}{\@title} \\}
    {\large \textit{\lname}, the language of \textit{???} \\}
    {\color{DeepSkyBlue3} \hprule \vspace{1.5ex} \\}
    % ----------------------------------------------------------------
    \vspace{1.5cm}
    {\Large\bfseries \@author}\\[5pt]
    %uruwi@protonmail.com\\[14pt]
    % ----------------------------------------------------------------
    \vspace{2cm}
    %{\Large\textlt{IVvoQMxeBPieLBxf TXxeKy}} \\
    \textnormal{een\^gs.-meibpelbe-kona} \\[5pt]
    \emph{A complete grammar}\\[2cm]
    %{in partial fulfilment for the award of the degree of} \\[2cm]
    %\tsc{\Large{{Doctor of Philosophy}}} \\[5pt]
    %{in some subject} \vspace{0.4cm} \\[2cm]
    % {By}\\[5pt] {\Large \sc {Me}}
    \vfill
    % ----------------------------------------------------------------
    %\includegraphics[width=0.19\textwidth]{example-image-a}\\[5pt]
    %{blah}\\[5pt]
    %{blahblah}\\[5pt]
    %{blahblah}\\
    \vfill
    {\@date}
  \end{center}
  \makeatother
\end{titlepage}

\pagecolor{LightSkyBlue2!15}

\begin{center}
    \textit{Dedicated to someone.}
\end{center}

\begin{verbatim}
Branch: canon
Version: 0.1
Date: 2018-04-14 (29 ful lax)
\end{verbatim}

(C)opyright 2018 Uruwi. See README.md for details.

\tableofcontents

\section{Introduction}

\chapter{Phonology and orthography}

\section{Phoneme inventory and (roman|hacm)isation}

Phonemes may have a noninitial or initial variant, or both. Initial phonemes are marked with a capital letter in both the romanisation and the hacmisation.

\begin{table}[h]
  \caption{Phonemes of \lname.}
  \centering
  \begin{tabular}{l|ll|l>{\kardinal}l}
    \# & NI & I & Roman & \textnormal{Hacm} \\
    \hline
    0 & m & & m & m \\
    27 & n & & n & n \\
    54 & ŋ & .ŋ & ŋ & n\^g \\
    162 & p & .p & p & p \\
    189 & & .t & t & t \\
    190 & ts & .ts & c & t\^s \\
    191 & tɬ & .tɬ & ċ & t\^x \\
    192 & tˤ & & ṭ & t\^a \\
    216 & k & & k & k \\
    217 & q & & q & k\^a \\
    324 & & .f & f & f \\
    350 & θ & .θx & þ & s\^f \\
    351 & s & .s & s & s \\
    352 & ɬ & & ṡ & x \\
    378 & x & & h & h \\
    379 & & .χ & ḥ & h\^a \\
    380 & xʷ & & w & w \\
    405 & ɹ & & r & r \\
    486 & a & .a & a & a \\
    513 & u & .u & u & u \\
  \end{tabular}
\end{table}

(In this document, we use the romanisation.)

Phoneme \#486 is an arbitrary open vowel, and \#513 is a closed or near-closed rounded vowel. Any other vowel may be inserted epenthetically.

The phoneme numbers listed are \emph{initiality-independent} (we shall call them \emph{inumbers}). \emph{Initiality-dependent} numbers (\emph{dnumbers}) are derived from the the former by leaving them as-is for non-initials and adding 13 for initials.

\section{Allophony}

The exact realisations of /u/ varies depending on the preceding phoneme:

\begin{table}[ht]
  \caption{Allophony of /u/.}
  \centering
  \begin{tabu}{l|Y}
    Allophone & Preceding \\
    \hline
    o & q χ \\
    u & ŋ tˤ k x xʷ ɹ \\
    ʉ & t ts s ɬ \\
    ʏ & θ tɬ n \\
    y & p f m \\
    ø & a \\
  \end{tabu}
\end{table}

\section{Phonotactics}

In \lname, a \emph{phonorun} consists of one initial phoneme followed by zero or more noninitial phonemes.

If a word begins with a non-initial phoneme, an initial vowel (usually /a/) is inserted at the front.

{}[TODO: example]

\subsection{Prosody}

In speech, a phonorun fits into an integral number of fixed-size \emph{cells}. The number of cells taken by a phonorun is roughly proportional to the number of vowels (including epenthetic vowels) pronounced. The last formal (non-epenthetic) vowel of a phonorun (if any) receives the stress.

\subsection{Syllabification}

{}[TODO: need some example sentences to come up with something useful]

A syllable contains a nucleus: one of the two formal vowels, an epenthetic vowel or a syllabic /ɹ/, in that order of preference.

Generally, syllables prefer not to cross phonorun boundaries, unless  $\rho_1$ ends with a vowel then a consonant, and $\rho_2$ begins with a vowel.

\chapter{Roots}

A root consists of three consonants (initial or otherwise). For any root $r$ (represented by a triplet of dnumbers), the following predicate $P$ holds for a permutation $s$ of $r$ if and only if $r = s$:

\begin{align*}
  P(a, b, c) &= L(A, B) \land L(B, C) & \text{where} \\
  L(p, q) &= ((q - p) \bmod 729) \le 364 \\
  u &= (a + b + c) \bmod 729 \\
  v &= (\min \{ w : w \ge v \land \gcd(w, 729) = 1 \}) \bmod 729 \\
  A &= (va + 128) \bmod 729 \\
  B &= (vb + 128) \bmod 729 \\
  C &= (vc + 128) \bmod 729
\end{align*}

In addition, a root has a gender of $(\mathbb{Z} \cap [-13, 13])^3$. This is used for adjectives and adverbs.

\chapter{Nouns}

Nouns are marked for one of the following cases:

\begin{table}[h]
  \caption{Cases of \lname.}
  \centering
  \begin{tabu}{l|l|Y}
    Case & Permutation & Explanation \\
    \hline
    Nominative & 123 & The subject of the sentence, as well as the possessor in a possessive phrase. \\
    Accusative & 132 & The direct object of the sentence. Also used for durations of time. \\
    Ablative & 213 & The origin of an action, either spatially or temporally. Also a vocative, instrumental or causal. \\
    Benefactive & 231 & An entity on whose behalf an action is done. \\
    Allative & 321 & The destination of an action, either spatially or temporally. The indirect object of the sentence (thus acting as a dative). Also a locative. \\
    Comitative & 312 & An entity in whose company an action is done. \\
  \end{tabu}
\end{table}

These other grammatical categories are marked:

\begin{itemize}
  \item Number-mutability: \emph{singular} (one object, and the quantity is unlikely to change), \emph{plural} (multiple, but the quantity does not change often) or \emph{mutable} (multiple, but the quantity changes often).
  \item Subjective attractiveness: \emph{neutral}, \emph{attractive} or \emph{unattractive}.
  \item Possession: if the noun is possessed, then it is marked for the person and NM of its possessor.
\end{itemize}

The schemata for number-mutability and subjective attractiveness is outlined in table \ref{table:declension1}.

\begin{table}[h]
  \caption{NM and attractiveness inflections in \lname.}
  \label{table:declension1}
  \centering
  \begin{tabular}{l|lll}
    NM \bs{} Attr & Neutral & Attractive & Unattractive \\
    \hline
    Singular & 1a23u & 12a3a & 1u2u3 \\
    Plural & 12a3ṡu & 1ṡa2a3 & ṡ1u23u \\
    Mutable & u12a3 & a1a23 & u1u23 \\
  \end{tabular}
\end{table}

The possessive affixes are outlined in table \ref{table:declension2}.

\begin{table}[h]
  \caption{Possessive affixes in \lname.}
  \label{table:declension2}
  \centering
  \begin{tabular}{l|lll}
    Person \bs{} NM & Singular & Plural & Mutable \\
    \hline
    1 & -wa & -Ḥa & -Ŋa \\
    2 & -pu & -Tu & -Ċu \\
    3(prox) & -kþ & -kṡ & -kh \\
    3(obv) & -qþ & -qṡ & -ra \\
  \end{tabular}
\end{table}

\appendix

\chapter{Dictionary}

An entry looks like this:

{}[placeholder]

From left to right:

\begin{enumerate}
    \item The entry -- the \lname{} term listed.
    \item The part of speech of the corresponding entry:
    \begin{itemize}
        \item \textit{r(\#, \#, \#)} -- a root of the specified gender
    \end{itemize}
    \item The definition -- the gloss for the corresponding entry.
    \begin{itemize}
        \item (S) -- nominative argument
        \item (P) -- absolutive argument
        \item (O) -- accusative argument
        \item (A) -- ergative argument
    \end{itemize}
    \item If applicable, any special grammatical or semantic notes for this term.
    \item Optionally, examples of usage.
\end{enumerate}

\begin{multicols}{2}
    \input{10/dict/dict.tex}
\end{multicols}

\end{document}