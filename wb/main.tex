\documentclass{book}

\usepackage[shortsuper,hacm,dhr]{common/uruwi}

\newcommand{\wname}{aaaaaaaaaaA}

\title{Compendium of Conworlds}
\author{uruwi}

\begin{document}

\pagecolor{Goldenrod!25}

\begin{titlepage}
    \makeatletter
    \begin{center}
        {\color{RedOrange} \hprule \vspace{1.5ex} \\}
        {\Huge \textcolor{BrickRed}{\@title}\\}
        %{\large \sffamily \textcolor{Purple}{\@title} \\}
        %{\large \textit{\lname}, the language of \textit{Rymako} \\}
        {\color{RedOrange} \hprule \vspace{1.5ex} \\}
        % ----------------------------------------------------------------
        \vspace{1.5cm}
        {\Large\bfseries \@author}\\[5pt]
        %uruwi@protonmail.com\\[14pt]
        % ----------------------------------------------------------------
        \vspace{2cm}
        %\textkardinal{aaaaaaaaaaaaaaaaa} \\
        %{aaaaaaaaaaaaaaaaa} \\[5pt]
        %\emph{A complete grammar}\\[2cm]
        %{in partial fulfilment for the award of the degree of} \\[2cm]
        %\tsc{\Large{{Doctor of Philosophy}}} \\[5pt]
        %{in some subject} \vspace{0.4cm} \\[2cm]
        % {By}\\[5pt] {\Large \sc {Me}}
        \vfill
        % ----------------------------------------------------------------
        %\includegraphics[width=0.19\textwidth]{example-image-a}\\[5pt]
        %{blah}\\[5pt]
        %{blahblah}\\[5pt]
        %{blahblah}\\
        \vfill
        {\@date}
    \end{center}
    \makeatother
\end{titlepage}

\pagecolor{Goldenrod!15}

\begin{verbatim}
Branch: canon
Version: 0.1
Date: 2018-03-01 (29 gil mel)
\end{verbatim}

(C)opyright 2018 Uruwi. See README.md for details.

\tableofcontents

\section{Introduction}

This document is a collection of worldbuilding-related articles that do not fit elsewhere. They are arranged in chronological order.

\chapter{Lkdċ}

\emph{Lkdċ} (Jbl: \hortho{lkdx}) are the functional equivalent of flags in \wname{}. They are sets of clothing used to identify nations (rather than being the day-to-day wear of their citizens). A wearer of such clothing will be referred to in this article by the Jbl term \emph{mwtgŋ} (\hortho{mwtgn\^g}).

\begin{table}
  \caption{Terms in various languages.}
  \centering
  \begin{tabular}{l|ll}
    Language & Outfit & Wearer \\
    \hline
    Jbl & \textkardinal{lkdx} & \textkardinal{mwtgn\^g} \\
    Varta Avina & rakaso & kasna \\
    Ḋraħýl Rase & lefkul & rikus \\
    levian9 & nikar & kuruþ \\
    \hline
    \emph{Lek-Tsaro} & \textkardinal{tci\^usa} & \textkardinal{n\^yu\^i.ara} \\
    \emph{Middle Rymakonian} & \textkardinal{tcqsa} (v3) & \textkardinal{n\^yu.aza} (v1) \\
  \end{tabular}
\end{table}

Unlike ordinary wear, the lkdċ is designed to be distinctive and a country's design will be precise with little room for variation. There are two main lkdċ systems: the \emph{Domain I system} and the \emph{Domain II/III system}, named after the regions that adopt them. Because of substantial differences between the two systems, they shall be addressed in separate sections. A system comprises of:

\begin{itemize}
  \item the guidelines for designing an lkdċ
  \item the guidelines for displaying an lkdċ
\end{itemize}

\section{The Domain I system}

\subsection{Design}

The following rules assume a height of at least 13 pivra and 5 nŷko (approx. 172.9~cm).

\emph{Mass:} The total mass of the lkdċ must not exceed 16 vyne (approx. 1.04~kg).

\emph{Volume:} The lkdċ must remain within a square 3 ṅitra (approx. 2.25~m) on each side, centred around the mwtgŋ, and must not exceed a height of 17 pivra (approx. 2.125~m).

\emph{Coverage:}

The lkdċ must be long enough to cover the knees, but not long enough to touch the ground.

We define the \emph{bottom of the shoulder} to be the point where the tangent to the shoulder (looking from the front) is at a $\pi/4$-radian angle from the ground. Then the part of the torso at least 2 [4.5] nŷko (approx. 4.16~cm [9.37~cm]) below the bottom of the shoulder at the front [back] must be covered.

The lkdċ must cover the head, but the face must be exposed.

\emph{Materials:} the precise materials used for the lkdċ are not specified, but rather only its appearance.

\emph{Colours:} at most five colours should be used (though the lkdċ of Nŋln uses 11). Traditionally, each article of clothing would use a single colour in order to reduce bleeding from washing, but this restriction is not followed as often in new designs. (Patterns other than solid colours are difficult to produce reliably and are discouraged.)

\subsection{Display}

The Domain I system distinguishes between \emph{live hanging} (on a person) and \emph{dead hanging} (on a frame)\footnote{In this sense, \emph{hanging} does not refer to a method of execution.}.

\subsubsection{Live hanging}

Live hanging is rarely -- if ever -- done for lkdċ that are not currently in use.

\emph{Preparation:} Both the lkdċ and the mwtgŋ must be washed thoroughly immediately before display.

\emph{Other tasks:} Performing another task while displaying is permitted, and more often than not this is the case.

However, there should be no objects obstructing the front of the mwtgŋ.

\emph{Display with other lkdċ:} When multiple lkdċ are displayed, the one of the home country is displayed at the ``centre'' (typically where the most attention is received) and the others on one side toward the periphery. Following that come the lkdċ of the subdivisions where they are displayed, ordered by descending size, and then the others, ordered by the smallest subdivision that the current location and the one represented by the lkdċ share (from smallest to largest).

\subsubsection{Dead hanging}

Dead hanging is used only for lkdċ that are not currently in use.

\emph{Preparation:} It is sometimes preferred to avoid washing the lkdċ (for archival reasons). In that case, washing it is not required.

\emph{Frame:} The frame must not resemble a human figure.

\subsection{Storage}

The lkdċ should be stored in a clean, dry, secure place when not in use.

\section{The Domain II/III rules}

TBD

\chapter{The ``progenitor of a language''}

A phenomenon prevalent in \wname{} is the so-called ``progenitor of a language'' -- a person who exerts a significant amount of influence on a language via standardisation and innovation. Such a figure is naturally in a position of power to impose such changes onto speakers of the language. In extreme cases, such as with Jbl, such a person might create a new language \emph{ex nihilo}. In almost all cases, however, the language is changed to the degree that mutual intelligibility with its parent language is lost.

In fact, languages are considered to exist on their own when it has a progenitor. For that reason, Middle Rymakonian is considered to be a separate language from Lek-Tsaro, but Modern Rymakonian is considered to be a continuation of its predecessor.

\begin{table}[h]
  \caption{Progenitors of languages. (Surnames are capitalised.)}
  \centering
  \begin{tabular}{l|ll}
    Language & Progenitor & Birth / Death \\
    \hline
    Jbl & Mjkdř w'HSHB (\textkardinal{mykdc w'hshb}) & 504 -- 582 \\
    Varta Avina & Satu VAHANU & 301 -- 354 \\
    Ḋraħýl Rase & \emph{none} & \\
    levian9 & FINUHAM Rtaṡiþ & 208? -- 273 \\
    \hline
    \emph{Lek-Tsaro} & Merhet (\textkardinal{merhet}) & Rukë 81 -- 144 \\
    \emph{Middle Rymakonian} & Ŝurak (\textkardinal{s\^wurak}) & Kûta 2168 -- Kasnepy 76 \\
  \end{tabular}
\end{table}

A notable exception to this archetype is Ḋraħýl Rase, which never had a progenitor; as a result, that language still resembles the languages of our world. Furthermore, due to the greatly accelerated rate of language change in the Frozen Gyre, the concept of a progenitor loses its meaning in that region.

\end{document}