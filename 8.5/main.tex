\documentclass{book}
  
\usepackage{common/uruwi}

\newcommand{\lname}{aaaaaaaaaaA}

\title{???}
\author{uruwi}

\begin{document}

\pagecolor{YellowOrange!25}

\begin{titlepage}
    \makeatletter
    \begin{center}
        %{\color{Orchid} \hprule \vspace{1.5ex} \\}
        {\Huge \sffamily \textcolor{Firebrick2}{\@title}\\}
        %{\large \kardinal \textcolor{Purple}{\@title} \\}
        {\large \textit{\lname}, the language of \textit{somewhere} \\}
        {\color{OrangeRed} \hprule \vspace{1.5ex} \\}
        % ----------------------------------------------------------------
        \vspace{1.5cm}
        {\Large\bfseries \@author}\\[5pt]
        %uruwi@protonmail.com\\[14pt]
        % ----------------------------------------------------------------
        \vspace{2cm}
        %\textkardinal{aaaaaaaaaaaaaaaaa} \\
        {aaaaaaaaaaaaaaaaa} \\[5pt]
        \emph{A complete grammar}\\[2cm]
        %{in partial fulfilment for the award of the degree of} \\[2cm]
        %\tsc{\Large{{Doctor of Philosophy}}} \\[5pt]
        %{in some subject} \vspace{0.4cm} \\[2cm]
        % {By}\\[5pt] {\Large \sc {Me}}
        \vfill
        % ----------------------------------------------------------------
        %\includegraphics[width=0.19\textwidth]{example-image-a}\\[5pt]
        %{blah}\\[5pt]
        %{blahblah}\\[5pt]
        %{blahblah}\\
        \vfill
        {\@date}
    \end{center}
    \makeatother
\end{titlepage}

\pagecolor{YellowOrange!15}

\begin{center}
    \textit{Dedicated to ostracod.}
\end{center}

\begin{verbatim}
Branch: canon
Version: 0.1
Date: 2017-11-24 (28 ruj nen)
\end{verbatim}

(C)opyright 2017 Uruwi. See README.md for details.

\tableofcontents

\section{Introduction}

\chapter{Phonology and orthography}

\section{Phonemes}

\lname{} has the following consonants:

\begin{table}[h]
  \caption{The consonants of \lname.}
  \centering
  \begin{tabular}{l|llllll}
      & Bilabial / & & & & & \\
      & Labiodental & Alveolar & Post-alveolar & Palatal & Velar & Glottal \\
      \hline
      Nasal & m & n & & ɲ & & \invalid \\
      Plosive & p p͈ b & t t͈ d & & & k k͈ ɡ & \\
      Fricative & f & s & tʃ t͈ʃ & & & h  \\
      Trill & & r & & & \invalid & \invalid \\
      Tap & & ɾ & & & \invalid & \invalid \\
      Approximant & & & & j & & \\
  \end{tabular}
\end{table}
\begin{table}[h]
  \centering
  \caption{The vowels of \lname.}
  \begin{tabular}{ll}
      Stressed & Unstressed \\
      \hline
      i & ɪ \\
      e & ə \\
      ɑ & æ \\
      o & ɔ \\
      u & ʊ \\
  \end{tabular}
\end{table}

\section{Phonotactics}

A syllable is allowed to consist of:

\begin{itemize}
    \item an onset, from one of:
    \begin{itemize}
        \item nothing at all
        \item a single consonant
        \item a non-tense plosive plus /ɾ/
        \item a plosive, fricative or nasal plus /j/
    \end{itemize}
    \item a vowel
    \item a coda, from one of /m n s ɾ/
\end{itemize}

\section{Allophony}

The following changes are made:

\begin{alignat*}{2}
  % \alpha &\rightarrow \omega &\quad(\lambda \blacklozenge \rho) &\quad[\Gamma]
  \text{h} &\rightarrow \text{x} &\quad(\blacklozenge \{\text{o}, \text{u}\}) \\
  \text{m} &\rightarrow \text{ɱ} &\quad(\blacklozenge \text{f}) \\
  \text{n} &\rightarrow \text{ŋ} &\quad(\blacklozenge C[+ve]) \\
  C_1[+v, +pl] &\rightarrow C_1[+fr] &\quad(V_1 \blacklozenge V_2) \\
  \{\text{i}, \text{e}, \text{ɑ}, \text{o}, \text{u}\}[-s]
  &\rightarrow \{\text{ɪ}, \text{ə}, \text{æ}, \text{ɔ}, \text{ʊ}\} &\quad \\
\end{alignat*}

\section{Orthography}

The language does not have a written form but we shall use the following romanisation:

\begin{table}[H]
  \caption{The consonants of \lname.}
  \centering
  \begin{tabular}{l|llllll}
      & Bilabial / & & & & & \\
      & Labiodental & Alveolar & Post-alveolar & Palatal & Velar & Glottal \\
      \hline
      Nasal & m & n & & ñ & & \invalid \\
      Plosive & p pp b & t tt d & & & k kk g & \\
      Fricative & f & s & ch cch & & & h  \\
      Trill & & rr & & & \invalid & \invalid \\
      Tap & & r & & & \invalid & \invalid \\
      Approximant & & & & y & & \\
  \end{tabular}
\end{table}
\begin{table}[H]
  \centering
  \caption{The vowels of \lname.}
  \begin{tabular}{l|ll}
      Rom & Stressed & Unstressed \\
      \hline
      i & i & ɪ \\
      e & e & ə \\
      a & ɑ & æ \\
      o & o & ɔ \\
      u & u & ʊ \\
  \end{tabular}
\end{table}

\ortho{r} represents /r/ at the start of a word but /ɾ/ elsewhere.

If the stress does not fall on the second-to-last syllable, then an acute accent appears on the stressed vowel.

\chapter{Syntax}

In this chapter, we look at the structure of the whole sentence.

\section{Basic word order}

\lname{} uses SVO order, although this is somewhat flexible due to verbal morphology.

\section{Modifiers}

Modifiers (adjectives and adverbs) follow what they modify.

\section{Questions}

In questions, the word order becomes VSO. If the primary verb is not auxiliary, then it requires support from the dummy auxiliary \ortho{kkir}.

\chapter{Honourifics}

\lname{} uses the following speech levels:

\begin{itemize}
  \item plain: used with unfamiliar or socially distant people, or when communicating to a general audience, or when mentioning a nonsentient entity.
  \item subservient: used toward superiors (incl. familial relationships ascending one or more generations).
  \item dominating: used toward inferiors (incl. familial relationships descending one or more generations).
  \item intimate: used between socially close people of similar status.
\end{itemize}

Note that there is a distinction between speaker-listener (SL) and speaker-target (ST) speech levels:

\begin{itemize}
  \item SL uses the speech level corresponding to the relationship between the speaker and the listener, and is used with first- and second-person pronouns, as well as verbs whose subjects are first- or second-person.
  \item ST uses the speech level corresponding to the relationship between the speaker and another entity (the \emph{topic}), and is used with third-person pronouns, as well as verbs whose subjects are third-person.
\end{itemize}

\chapter{Nouns}

Nouns fall into one of two genders and are marked for definiteness.

\section{Gender}

\lname{} has two genders: \emph{feminine} and \emph{masculine}.

Nouns tend to be feminine if they end in one of the following:

\begin{itemize}
  \item \ortho{-a}
  \item \ortho{-in}
  \item \ortho{-er}
  \item \ortho{-ir}
\end{itemize}

Nouns tend to be masculine if the end in one of the following:

\begin{itemize}
  \item \ortho{-o}
  \item \ortho{-u}
  \item \ortho{-as}
  \item \ortho{-os}
  \item \ortho{-us}
  \item \ortho{-ur}
\end{itemize}

These are only tendencies -- for instance, \ortho{mora} is masculine and \ortho{kas} is feminine.

A group of objects of both the feminine and masculine genders is regarded as feminine.

\section{Definiteness}

There are three degrees of definiteness in \lname:

\begin{itemize}
  \item Indefinite: the referent is not identifiable.
  \item Definite: the referent is identifiable. Unlike in English, names fall under this category.
  \item Generic: refers to the idea of something, rather than the entity itself, or a general statement: \\
  ~\\
  \hli{Te} \hlii{mepúr} \hliii{suma} \hliv{kkara.} \\
  \hli{\tsc{def.fem}} \hlii{fish} \hliii{eat-\tsc{3.pres.imp.ind}} \hliv{flower} \\
  \hli{The} \hlii{fish} \hliii{eats} \hliv{flowers.} \\
  ~\\
  \hli{Che} \hlii{mepúr} \hliii{suma} \hliv{kkara.} \\
  \hli{\tsc{gen.fem}} \hlii{fish} \hliii{eat-\tsc{3.pres.imp.ind}} \hliv{flower} \\
  \hlii{Fish} \hliii{eat} \hliv{flowers.}
\end{itemize}

\section{Articles}

There are definite and generic articles of each gender; indefinite noun phrases receive no article. Before a vowel, articles become clitics.

\begin{table}[h]
  \caption{Articles in \lname.}
  \centering
  \begin{tabular}{l|ll}
    & Feminine & Masculine \\
    \hline
    Definite & te (t') & tu (t') \\
    Generic & che (ch') & ho (h') \\
  \end{tabular}
\end{table}

Note that articles occur before the entire noun phrase, not the noun itself.

\section{Pronouns}

\subsection{Personal pronouns}

Personal pronouns are separated not only by gender, but also by speech style.

\begin{longtable}[c]{r|ll|ll|ll}
  \caption{Personal pronouns.} \\

  & \multicolumn{2}{c|}{Plain / Dom} & \multicolumn{2}{c|}{Subservient} & \multicolumn{2}{c}{Intimate}\\
  P & fem & masc & fem & masc & fem & masc \\
  \hline
  \endfirsthead

  & \multicolumn{2}{c|}{Plain / Dom} & \multicolumn{2}{c|}{Subservient} & \multicolumn{2}{c}{Intimate}\\
  P & fem & masc & fem & masc & fem & masc \\
  \hline
  \endhead
  
  \endfoot
  
  \endlastfoot
  
  \multicolumn{7}{c}{Nominative} \\
  \hline
  1 & na & nu & han & dan & \multicolumn{2}{c}{bi} \\
  2 & \multicolumn{2}{c|}{te} & \invalid & \invalid & isi & isu \\
  3 & sa & so & sa & so & cha & chu \\
  \hline
  \multicolumn{7}{c}{Genitive} \\
  \hline
  1 & nas & nus & hanis & danis & \multicolumn{2}{c}{bis} \\
  2 & \multicolumn{2}{c|}{tes} & \invalid & \invalid & iris & irus \\
  3 & sam & som & sam & som & cham & chum \\
  \hline
  \multicolumn{7}{c}{Accusative} \\
  \hline
  1 & \multicolumn{2}{c|}{ke (k')} & \multicolumn{2}{c|}{pe (p')} & \multicolumn{2}{c}{mi (m')} \\
  2 & \multicolumn{2}{c|}{ti (tt')} & \multicolumn{2}{c|}{si (s')} & isi & isu \\
  3 & \multicolumn{4}{c|}{si (s')} & \multicolumn{2}{c}{chi (cch')} \\
  \hline
  \multicolumn{7}{c}{Dative} \\
  \hline
  1 & \multicolumn{2}{c|}{ko} & \multicolumn{2}{c|}{pu} & \multicolumn{2}{c}{be} \\
  2 & \multicolumn{2}{c|}{tir} & \multicolumn{2}{c|}{ser} & isir & isur \\
  3 & \multicolumn{4}{c|}{ser} & \multicolumn{2}{c}{cher} \\
\end{longtable}

Second-person nominative and genitive pronouns are not used in the subservient speech style, but rather the listener's title.

The dative and accusative pronouns fall right before the verb, in that order. If the verb starts with a vowel, then the clitic (in parentheses, if applicable) is used in place of the usual accusative pronoun. The dative pronoun \emph{must be used} if there is an indirect object, even if it is stated explicitly somewhere else in the sentence.

Again, genitive pronouns occur before the entire noun phrase, not the noun itself.

\subsection{Correlatives}

Determiners always precede their antecedents, and they are categorised into indefinite and definite quantifiers.

\begin{table}[ht]
  \caption{Determiners.}
  \centering
  \begin{tabular}{l|l}
    Indefinite & Definite \\
    \hline
    mipe \emph{any, some, either} & teba / tebu \emph{this} \\
    sita / situ \emph{many} & treba / trebu \emph{that} \\
    ttyen \emph{no, none} & hekka \emph{all, every} \\
    kike \emph{few, little} & \\
    fine \emph{what} & \\
  \end{tabular}
\end{table}

\ortho{kade} \emph{another, other} can be either indefinite or definite.

A noun phrase modified by a definite quantifier must receive a definite article as well: \ortho{hekka te mepúr} \emph{every fish}. Similarly, correlatives inherit the definiteness of their parent determiners and likewise receive appropriate articles.

Correlatives are shown in table \ref{table:indefinite}.

\footnotesize
\newgeometry{margin=1cm}
%\begin{landscape}
  \begin{longtabu} to \textwidth {l|lllllllll}
    \caption{Indefinite pronouns.} \label{table:indefinite} \\

    %& interrogative & proximal & distal & existential & universal & negatory & alternative & & \\
    & what & this & that & some, any & all & none & other & much, many & few, little \\
    \hline
    \endfirsthead

    %& interrogative & proximal & distal & existential & universal & negatory & alternative & & \\
    & what & this & that & some, any & all & none & other & much, many & few, little \\
    \hline
    \endhead
    
    \endfoot
    
    \endlastfoot

    \rowfont{\rmfamily} determiner & fine & teba/u & treba/u & mipe & hekka & ttyen & kade & sita/u & kike \\
    \rowfont{\itshape} & what, which & this & that & some, any & all & no & another & much, many & few, little \\
    \hline
    \rowfont{\rmfamily} pronoun (nonhuman) & fin & kun & fon & miba & heya & ttyan & kadin & sitan & \invalid \\
    \rowfont{\itshape} & what & this one & that one & something, anything & everything & nothing & something else & many things & \invalid \\
    \rowfont{\rmfamily} pronoun (human) & inas & kesti & sesti & mimba & ginda & ttiña & kanda & siten & \invalid \\
    \rowfont{\itshape} & who & this person & that person & someone, anyone & everyone & no one & someone else & many people & \invalid \\
    \rowfont{\rmfamily} pronoun (out of N) & kres & imta & imtra & mimpa & hemka & ttyem & kambu & \invalid & \invalid \\
    \rowfont{\itshape} & which & this one & that one & some, whichever & all, each & none & other, another & \invalid & \invalid \\
    \hline
    \rowfont{\rmfamily} location & ribe & egas & artá & chas & resím & ttunu & akada & nesin & kirpo \\
    \rowfont{\itshape} & where & here & there & somewhere & everywhere & nowhere & elsewhere & in many places & in few places \\
    \rowfont{\rmfamily} time & borre & cchen & yan & depe & resim & mettu & akeda & nemis & kikiri \\
    \rowfont{\itshape} & when & now & then & anytime & always & never & another time & often & seldom \\
    \rowfont{\rmfamily} reason & nine & kkere & \invalid & mirba & \invalid & mettu & \invalid & \invalid & kiraba \\
    \rowfont{\itshape} & why & this reason & \invalid & some reason & \invalid & no reason & \invalid & \invalid & few reasons \\
  \end{longtabu}
%\end{landscape}
\restoregeometry
\normalsize

\chapter{Verbs}

Verbs in \lname{} are conjugated according to:

\begin{itemize}
  \item Person of the subject (1st, 2nd, 3rd)
  \item Tense (past, present, future)
  \item Aspect (imperfect, perfect)
  \item Mood (indicative, subjunctive)
\end{itemize}

\begin{longtable}[c]{r|llll}
  \caption{Verb conjugations for verbs whose infinitives end in \ortho{-is}.} \\
  
  P & II & IP & SI & SP \\
  \endfirsthead

  P & II & IP & SI & SP \\
  \hline
  \endhead

  \endfoot

  \endlastfoot

  \hline
  \multicolumn{5}{c}{Present} \\
  \nobreakmidrule
  1 & -e & -es & -eda & -esta \\
  2 & -u & -us & -utta & -usta \\
  3 & -a & -an & -asa & -asta \\
  \hline
  \multicolumn{5}{c}{Past} \\
  \nobreakmidrule
  1 & -i & -is & -ita & -ista \\
  2 & -uka & -uska & -ukka & -uska \\
  3 & -o & -os & -ona & -onsa \\
  \hline
  \multicolumn{5}{c}{Future} \\
  \nobreakmidrule
  1 & -ore & -ose & -orda & -ordas \\
  2 & -oru & -oso & -ortta & -ostra \\
  3 & -ora & -osa & -orna & -onsa \\
\end{longtable}

\begin{longtable}[c]{r|llll}
  \caption{Verb conjugations for verbs whose infinitives end in \ortho{-ir}.} \\
  
  P & II & IP & SI & SP \\
  \endfirsthead

  P & II & IP & SI & SP \\
  \hline
  \endhead

  \endfoot

  \endlastfoot

  \hline
  \multicolumn{5}{c}{Present} \\
  \nobreakmidrule
  1 & -e & -er & -eda & -erta \\
  2 & -u & -ur & -utta & -urta \\
  3 & -a & -an & -asa & -arsa \\
  \hline
  \multicolumn{5}{c}{Past} \\
  \nobreakmidrule
  1 & -i & -ir & -ita & -itra \\
  2 & -uka & -urka & -ukka & -ukra \\
  3 & -o & -or & -ona & -orna \\
  \hline
  \multicolumn{5}{c}{Future} \\
  \nobreakmidrule
  1 & -ose & -ore & -osta & -ostar \\
  2 & -osu & -oro & -ostra & -otra \\
  3 & -osa & -ora & -onsa & -orna \\
\end{longtable}

\begin{longtable}[c]{r|llll}
  \caption{Verb conjugations for \ortho{fir} \emph{be\textsuperscript{1}}.} \\
  
  P & II & IP & SI & SP \\
  \endfirsthead

  P & II & IP & SI & SP \\
  \hline
  \endhead

  \endfoot

  \endlastfoot

  \hline
  \multicolumn{5}{c}{Present} \\
  \nobreakmidrule
  1 & fie & fyer & kkeda & kkerta \\
  2 & fyu & fyur & kkutta & kkurta \\
  3 & fia & fian & kkasa & kkarsa \\
  \hline
  \multicolumn{5}{c}{Past} \\
  \nobreakmidrule
  1 & ñi & ñir & ñita & ñitra \\
  2 & fuka & furka & fokka & fukra \\
  3 & fio & for & fona & forna \\
  \hline
  \multicolumn{5}{c}{Future} \\
  \nobreakmidrule
  1 & fes & ferus & kkosta & kkostar \\
  2 & fuas & furus & kkostra & kkotra \\
  3 & fas & farus & kkonsa & kkorna \\
\end{longtable}

\begin{longtable}[c]{r|llll}
  \caption{Verb conjugations for \ortho{abis} \emph{be\textsuperscript{2}}.} \\
  
  P & II & IP & SI & SP \\
  \endfirsthead

  P & II & IP & SI & SP \\
  \hline
  \endhead

  \endfoot

  \endlastfoot

  \hline
  \multicolumn{5}{c}{Present} \\
  \nobreakmidrule
  1 & ppe & abes & ppeda & abesta \\
  2 & ppu & abus & pputta & abusta \\
  3 & ppa & aban & pasa & pasta \\
  \hline
  \multicolumn{5}{c}{Past} \\
  \nobreakmidrule
  1 & abi & gis & abita & abista \\
  2 & abuka & guska & ppukka & abuska \\
  3 & abo & abos & pona & ponsa \\
  \hline
  \multicolumn{5}{c}{Future} \\
  \nobreakmidrule
  1 & gore & guse & morda & abordas \\
  2 & goru & guso & mortta & abostra \\
  3 & gora & gusa & porna & ponsa \\
\end{longtable}

\begin{longtable}[c]{r|llll}
  \caption{Verb conjugations for \ortho{nis} \emph{be able to}.} \\
  
  P & II & IP & SI & SP \\
  \endfirsthead

  P & II & IP & SI & SP \\
  \hline
  \endhead

  \endfoot

  \endlastfoot
  
  \hline
  \multicolumn{5}{c}{Present} \\
  \nobreakmidrule
  1 & nie & ner & neda & nerta \\
  2 & niu & nur & nutta & nurta \\
  3 & nia & nan & nasa & narsa \\
  \hline
  \multicolumn{5}{c}{Past} \\
  \nobreakmidrule
  1 & nio & nir & nita & nitra \\
  2 & nuka & nurka & nukka & nukra \\
  3 & nio & nor & noda & norna \\
  \hline
  \multicolumn{5}{c}{Future} \\
  \nobreakmidrule
  1 & nose & nore & nosta & nostar \\
  2 & nosu & noro & nostra & notra \\
  3 & nosa & nora & nonsa & norna \\
\end{longtable}

\section{Imperatives and volitionals}

\begin{longtable}[c]{l|llll}
  \caption{Imperative and volitional forms for verbs.} \\
  
  & Plain & Intimate & Dominating & Subservient \\
  \endfirsthead

  & Plain & Intimate & Dominating & Subservient \\
  \hline
  \endhead

  \endfoot

  \endlastfoot

  \hline
  \multicolumn{5}{c}{Regular form} \\
  \nobreakmidrule
  Imp-I & fo (2PresSI) & -o & -o & respe fo (2PresSI) \\
  Imp-P & fo (2PresSP) & -os & -os & respe fo (2PresSP) \\
  Vol-I & fo (1FutSI) & -ono & -ano & respore fo (1FutSI) \\
  Vol-P & fo (1FutSP) & -onos & -anos & respore fo (1FutSP) \\
  \hline
  \multicolumn{5}{c}{\ortho{fir} \emph{be\textsuperscript{1}}} \\
  \nobreakmidrule
  Imp-I & fo kkutta & fyo & fio & respe fo kkutta \\
  Imp-P & fo kkurta & fyos & fios & respe fo kkurta \\
  Vol-I & fo kkosta & fyon & fian & respore fo kkosta \\
  Vol-P & fo kkostar & fyonos & fias & respore fo kkostar \\
  \hline
  \multicolumn{5}{c}{\ortho{abir} \emph{be\textsuperscript{2}}} \\
  \nobreakmidrule
  Imp-I & fo pputta & ppo & ppo & respe fo pputta \\
  Imp-P & fo abusta & ppos & ppos & respe fo abusta \\
  Vol-I & fo mortta & ppon & ppan & respore fo mortta \\
  Vol-P & fo abordas & ppos & ppas & respore fo abordas \\
  \hline
  \multicolumn{5}{c}{\ortho{nir} \emph{be able to}} \\
  \nobreakmidrule
  Imp-I & fo nutta & nio & nio & respe fo nutta \\
  Imp-P & fo nurta & nos & nos & respe fo nurta \\
  Vol-I & fo nosta & yono & nano & respore fo nosta \\
  Vol-P & fo nostar & yonos & nanos & respore fo nostar \\
\end{longtable}

\section{Present and past participles}

\begin{table}[h]
  \caption{Present and past participles.}
  \centering
  \begin{tabular}{r|ll}
    Verb & Present & Past \\
    \hline
    Ending with \ortho{-is} & -inde & -isti \\
    Ending with \ortho{-ir} & -irne & -irri \\
    \ortho{fir} & kkade & fi \\
    \ortho{abir} & aben & abri \\
  \end{tabular}
\end{table}

\ortho{abir} can be combined with the present participle to form the present progressive, and with the past participle to form the passive form.

\section{Honourifics}

Some verbs have multiple forms depending on the speech level used.

\section{Auxiliary verbs}

Auxiliary verbs are conjugated and placed before the main verb, which is put in the infinitive form: \\
~\\
\hli{Mida} \hlii{kkibis.} \\
\hli{want-3.\tsc{pres.ind.imp}} \hlii{eat-\tsc{inf}} \\
\hli{They want to} \hlii{eat.} \\

Some auxiliary verbs require prepositions, or adopt a different meaning when a preposition is used.

\section{Negation}

Only auxiliary verbs can be negated directly: \\
~\\
\hli{Ttya} \hlii{nie} \hliii{kapir.} \\
\hli{\tsc{neg}} \hlii{be\_able\_to-1.\tsc{pres.ind.imp}} \hliii{come-\tsc{inf}} \\
\hlii{I can}\hli{'t} \hliii{come.} \\

Other verbs require the dummy auxiliary \ortho{kkir}: \\
~\\
\hli{Ttya} \hlii{kkor} \hliii{kapir.} \\
\hli{\tsc{neg}} \hlii{do-3.\tsc{past.ind.perf}} \hliii{come-\tsc{inf}} \\
\hlii{He didn}\hli{'t} \hliii{come.}

\chapter{Adjectives}

Adjectives usually follow the nouns they modify, but may occasionally precede their antecedents for emphasis or such. They are declined for gender.

\section{One-form adjectives}

One-form adjectives are not inflected at all, and are more likely to precede their antecedents than two-form adjectives.

\section{Two-form adjectives}

Two-form adjectives are inflected for gender.

\begin{table}[h]
  \caption{Declension of two-form adjectives.}
  \centering
  \begin{tabular}{ll}
    Fem & Masc \\
    \hline
    -a & -u \\
    -is & -as \\
    -ir & -ur \\
    -er & -ur \\
    -in & -o \\
  \end{tabular}
\end{table}

\section{Adverbs}

Adverbs are formed from the feminine forms of adjectives by appending \ortho{-ta}.

\section{Numerals}

\lname{} has a restricted numeral system:

\begin{itemize}
  \item \ortho{men} \emph{one}
  \item \ortho{fua} \emph{two}
  \item \ortho{bitta} \emph{three}
\end{itemize}

\chapter{Prepositions}

The most common prepositions in \lname{} are:

\begin{table}[ht]
  \caption{Prepositions.}
  \centering
  \begin{tabu} to \linewidth {l|Y}
    Preposition & Translation \\
    \hline
    us & of, from, originating from \\
    usta & since \\
    fe & to, toward, until \\
    ten & at, in, inside \\
    hata & with (both comitative and instrumental) \\
  \end{tabu}
\end{table}

\appendix

\chapter{Dictionary}

An entry looks like this:

\textsf{magis} \textit{vt}
\quad (S) throws (O)

From left to right:

\begin{enumerate}
    \item The entry -- the \lname{} term listed.
    \item The part of speech of the corresponding entry:
    \begin{itemize}
      \item \textit{nf} -- a feminine noun
      \item \textit{nm} -- a masculine noun
      \item \textit{vt} -- a transitive verb
      \item \textit{vi} -- an intransitive verb
      \item \textit{va} -- an ambitransitive verb
      \item \textit{vh} -- an auxiliary verb
      \item \textit{a1} -- a one-form adjective
      \item \textit{a2} -- a two-form adjective (feminine form listed)
    \end{itemize}
    \item The definition -- the gloss for the corresponding entry.
    \begin{enumerate}
        \item (S) -- the subject
        \item (O) -- the direct object
        \item (I) -- the indirect object
    \end{enumerate}
    \item If applicable, any special grammatical or semantic notes for this term.
    \item Optionally, examples of usage.
\end{enumerate}

\begin{multicols}{2}
    \input{8.5/dict/dict.tex}
\end{multicols}

\end{document}