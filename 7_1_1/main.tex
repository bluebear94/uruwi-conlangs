\documentclass{book}

\usepackage[shortsuper,hacm,ltfont,nedham]{common/uruwi}

\newcommand{\lname}{Modern Rymakonian}
\newcommand{\northo}[1]{\ortho{\textnedham{#1}}}

\title{lek-\bs{}rqmako-ma lek-ma st\^xee-\bs{}rqmako}
\author{uruwi}

\begin{document}

\pagecolor{Thistle!25}

\begin{titlepage}
    \makeatletter
    \begin{center}
        {\color{Orchid} \hprule \vspace{1.5ex} \\}
        %{\Huge \ltfont \textcolor{Plum}{LKe\bs{}TSxaRMoa SL LKeMa SL STfXe\bs{}RMyaKo}\\}
        {\Huge \dxvi \textcolor{Purple}{\@title} \\}
        {\large \textit{\lname}, the language of \textit{Rymako} \\}
        {\color{Orchid} \hprule \vspace{1.5ex} \\}
        % ----------------------------------------------------------------
        \vspace{1.5cm}
        {\Large\bfseries \@author}\\[5pt]
        %uruwi@protonmail.com\\[14pt]
        % ----------------------------------------------------------------
        \vspace{2cm}
        %{\Large\textlt{IVvoQMxeBPieLBxf TXxeKy}} \\
        \textkardinal{een\^gs.-meibpelbe-kona} \\[5pt]
        \emph{A complete grammar}\\[2cm]
        %{in partial fulfilment for the award of the degree of} \\[2cm]
        %\tsc{\Large{{Doctor of Philosophy}}} \\[5pt]
        %{in some subject} \vspace{0.4cm} \\[2cm]
        % {By}\\[5pt] {\Large \sc {Me}}
        \vfill
        % ----------------------------------------------------------------
        %\includegraphics[width=0.19\textwidth]{example-image-a}\\[5pt]
        %{blah}\\[5pt]
        %{blahblah}\\[5pt]
        %{blahblah}\\
        \vfill
        {\@date}
    \end{center}
    \makeatother
\end{titlepage}

\pagecolor{Thistle!15}

\begin{center}
    \textit{Dedicated to Gufferdk.}
\end{center}

\begin{verbatim}
Branch: canon
Version: 0.1
Date: 2018-04-30 (29 dyu mik)
\end{verbatim}

(C)opyright 2018 Uruwi. See README.md for details.

\tableofcontents

\section{Introduction}

\chapter{Phonology and orthography}

\section{Diachronics}

\subsection{L-recession: Kasnepy 430 -- 490}

The first set of sound changes involves the loss of /l/.

For nouns, this proceeds as follows:

\begin{alignat*}{2}
  % \alpha &\rightarrow \omega &\quad(\lambda \blacklozenge \rho) &\quad[\Gamma]
  \text{ʔ} &\rightarrow \emptyset \\
  C_1[+ap] \text{l} &\rightarrow C_1 \\
  C_1[+na] \text{l} &\rightarrow \text{ŋ} \\
  \text{ɾl} &\rightarrow \text{r} \\
  C_1[+lb] \text{l} &\rightarrow C_1[+velarised] \\
  C_1[+ve] \text{l} &\rightarrow C_1 \\
  C_1[+lf] \text{l} &\rightarrow C_1 \\
  C_1[+whistled] \text{l} &\rightarrow C_1[+lf, -whistled] \\
  C_1[+affricate] \text{l} &\rightarrow C_1 \\
  C_1 \text{l} &\rightarrow C_1[+pharyngealised] \\
  V_1 \text{l} &\rightarrow V_1[+lateralised] \\
  \text{l} &\rightarrow \text{dˤ}  &\quad(\square \blacklozenge) \\
  \{\text{ʌ}, \text{ɯ}\} &\rightarrow \{\text{o}, \text{u}\} &\quad \lnot(\text{w} \blacklozenge) \\
  \text{ɯ} &\rightarrow \text{i}
\end{alignat*}

For verbs, these changes occur instead:

\begin{alignat*}{2}
  % \alpha &\rightarrow \omega &\quad(\lambda \blacklozenge \rho) &\quad[\Gamma]
  \text{ʔ} &\rightarrow \text{k} \\
  \text{l} C_1[+ap] &\rightarrow C_1 \\
  \text{l} C_1[+na] &\rightarrow \text{ŋ} \\
  \text{lɾ} &\rightarrow \text{r} \\
  \text{l} C_1[+lb] &\rightarrow C_1[+velarised] \\
  \text{l} C_1[+ve] &\rightarrow C_1 \\
  \text{l} C_1[+lf] &\rightarrow C_1 \\
  \text{l} C_1[+whistled] &\rightarrow C_1[+lf, -whistled] \\
  \text{l} C_1[+affricate] &\rightarrow C_1 \\
  \text{l} C_1 &\rightarrow C_1[+pharyngealised] \\
  \text{l} V_1 &\rightarrow V_1 \text{z} \\
  \text{l} &\rightarrow \text{t}  &\quad(\blacklozenge \square) \\
  \{\text{ʌ}, \text{ɯ}\} &\rightarrow \{\text{e}, \text{i}\} \\
\end{alignat*}

Of course, making sound changes depend on a word's part of speech is a cardinal sin of diachronics, but since when were linguistic universals a concern?

(The observant reader might notice the short timespan of these changes. This is not an error.)

\subsection{Vocaloëxodus: Kasnepy 660 -- Nihel 50}

At this point, vowels start to be lost. The first one to be lost is /ʉ̜/:

\begin{alignat*}{2}
  % \alpha &\rightarrow \omega &\quad(\lambda \blacklozenge \rho) &\quad[\Gamma]
  \text{jʉ̜} &\rightarrow \text{i} \\
  \text{ʉ̜} &\rightarrow \emptyset &\quad(\blacklozenge \square \lor \square \blacklozenge) \\
  \text{ʉ̜} &\rightarrow \text{u} &\quad(C_1[+lb] \blacklozenge) \\
  \text{wu} &\rightarrow \text{u} \\
  \text{ʉ̜} &\rightarrow \text{o} &\quad(C_1[+ve] \blacklozenge) \\
  \text{ʉ̜} &\rightarrow \text{e} &\quad(C_1 \blacklozenge) \\
  \text{ʉ̜} &\rightarrow \emptyset \\
  \text{ʉ̜ˡ} &\rightarrow \text{ɹ}
\end{alignat*}

This is followed by vowel merging:

\begin{alignat*}{2}
  % \alpha &\rightarrow \omega &\quad(\lambda \blacklozenge \rho) &\quad[\Gamma]
  V_1V_1 &\rightarrow V_1[+l] \\
  \text{i} V_1 &\rightarrow \text{j} V_1 \\
  \text{u} V_1 &\rightarrow \text{w} V_1
\end{alignat*}

After this change, \emph{lateral rotation} takes place: lateralisation transfers from one vowel to the next within a word, wrapping back to the first vowel from the last. Thus, /toˡu/ becomes /touˡ/ -- the lateralisation transfers from the first vowel to the second.

(Short) /u/ is the next vowel to be lost:

\begin{alignat*}{2}
  % \alpha &\rightarrow \omega &\quad(\lambda \blacklozenge \rho) &\quad[\Gamma]
  \{\text{u}, \text{uˡ}\} &\rightarrow \emptyset &\quad(\square \blacklozenge) \\
  \{\text{u}, \text{uˡ}\} &\rightarrow \text{v} &\quad(V_1 \blacklozenge) \\
  C_1\{\text{k}, \text{ɡ}, \text{x}, \text{ɣ}, \text{ŋ}\} V_1\{\text{u}, \text{uˡ}\} &\rightarrow
    C_1\{\text{p}, \text{b}, \text{f}, \text{v}, \text{m}\} V_1\{\text{a}, \text{aˡ}\} \\
  C_1\{\text{fx}, \text{vɣ}, \text{θx}, \text{ðɣ}\} V_1\{\text{u}, \text{uˡ}\} &\rightarrow
    C_1\{\text{f}, \text{v}, \text{θ̼}, \text{ð̼}\} V_1\{\text{e}, \text{eˡ}\} \\
  C_1\{\text{s}, \text{z}, \text{ʃ}, \text{ʒ}\} \text{u} &\rightarrow
    C_1\{\text{s͎}, \text{z͎}, \text{ʃ͎}, \text{ʒ͎}\} \text{e} \\
  C_1\{\text{s}, \text{z}, \text{ʃ}, \text{ʒ}\} \text{uˡ} &\rightarrow
    C_1\{\text{ɬ}, \text{ɮ}, \text{ɬ}, \text{ɮ}\} \text{e} \\
  \text{u} &\rightarrow \emptyset \\
  \text{uˡ} &\rightarrow \text{ɹ} \\
  \text{w} &\rightarrow \text{v}
\end{alignat*}

After /u/, /i e ʌ/ (and their lateral counterparts) are lost:

\begin{alignat*}{2}
  % \alpha &\rightarrow \omega &\quad(\lambda \blacklozenge \rho) &\quad[\Gamma]
  \{\text{i}, \text{e}, \text{ʌ}\} &\rightarrow
    \emptyset &\quad(\blacklozenge \square) \\
  \{\text{iˡ}, \text{eˡ}, \text{ʌˡ}\} &\rightarrow
    \{\text{l}, \text{l}, \text{ɫ}\} \\
  \{\text{i}, \text{e}, \text{ʌ}\} C_1 &\rightarrow
    \emptyset &\quad(C_1 \blacklozenge) &\quad[\#\delta > 3] \\
  \text{e} &\rightarrow \emptyset &\quad(C_1[+whistled] \blacklozenge) \\
  \{\text{i}, \text{e}, \text{ʌ}\} &\rightarrow
    \{\text{ʃ}, \text{s}, \text{θx}\}
\end{alignat*}

/o/ is the next vowel to be lost:

\begin{alignat*}{2}
  % \alpha &\rightarrow \omega &\quad(\lambda \blacklozenge \rho) &\quad[\Gamma]
  \text{oˡ} &\rightarrow \text{ɫ} \\
  C_1[+na] \text{o} &\rightarrow C_1[+nareal] &\quad(\blacklozenge \square) \\
  \{\text{p}, \text{t}, \text{c}\} \text{o} &\rightarrow
    \{\text{ʘ}, \text{ǀ}, \text{ǂ}\} \\
  \{\text{b}, \text{d}, \text{ɟ}, \text{ɡ}\} \text{o} &\rightarrow
    \{\text{p}, \text{t}, \text{c}, \text{k}\} \\
  \{\text{m}, \text{n}, \text{ɲ}, \text{ŋ}\} \text{o} &\rightarrow
    \{\text{b}, \text{d}, \text{ɟ}, \text{ɡ}\} \\
  \{\text{f}, \text{v}\} \text{o} &\rightarrow \text{p} &\quad(\blacklozenge \{\square, C_2[+ap], C_2[+la]\}) \\
  \{\text{θ}, \text{ð}, \text{s}, \text{z}, \text{s͎}, \text{z͎}\} \text{o} &\rightarrow \text{t} &\quad(\ldots) \\
  \{\text{ʃ}, \text{ʒ}, \text{ʃ͎}, \text{ʒ͎}, \text{x}, \text{ɣ}\} \text{o} &\rightarrow \text{k} &\quad(\ldots) \\
  \{\text{f}, \text{v}, \text{θ}, \text{ð}\} \text{o} &\rightarrow
    \{\text{ʕ}, \text{ħ}, \text{ʁ}, \text{χ}\} \\
  \text{o} &\rightarrow C_1[+fr] &\quad(\square \blacklozenge C_1[+pl]) \\
  \text{o} &\rightarrow \text{ɣ}
\end{alignat*}

Finally /a/ is lost: $\{\text{a}, \text{aˡ}\} \rightarrow \emptyset$. The long vowels can subsequently be reänalysed as being short.

It should be noted that epenthetic vowels are allowed between consonants.

\subsection{Cluster reduction: Nihel 70 -- 130}

The consonant clusters resulting from the previous vocaloëxodus turn out to be quite complex. Let $f$ be as such:

\begin{align*}
  f(p) &=
  \begin{cases}
    3 & p \text{ is voiced or pharyngealised} \\
    2 & p = \text{k} \text{ or $p$ is velarised} \\
    1 & p = \text{t} \\
    0 & p \in \{\text{p}, \text{c}\}
  \end{cases}
\end{align*}

Then

\begin{alignat*}{2}
  % \alpha &\rightarrow \omega &\quad(\lambda \blacklozenge \rho) &\quad[\Gamma]
  C_1[+pl] C_2[+pl] &\rightarrow C_1 &\quad \lnot(V_1 \blacklozenge V_2) &\quad[f(C_1) \ge f(C_2)] \\
  C_1[+pl] C_2[+pl] &\rightarrow C_2 &\quad \lnot(V_1 \blacklozenge V_2) &\quad[f(C_2) > f(C_1)] \\
  C_1\{\text{l}, \text{ɫ}, \text{ɹ}\} C_2[+pl] &\rightarrow C_2 C_1 &\quad(C_3 \blacklozenge) \\
  C_1[+na] &\rightarrow C_1[pa=x] &\quad(\blacklozenge C_2[-na, -nareal,& -lateral, pa=x]) \\
  C_1[+click] C_2[+pl, -ve, -v] &\rightarrow C_2[+click] \\
  C_1[+pl, -ve, -v] C_2[+click] &\rightarrow C_1[+click] \\
  C_1[+click] C_2[+pl, -ve, +v] &\rightarrow C_2[+implosive] \\
  C_1[+pl, -ve, +v] C_2[+click] &\rightarrow C_1[+implosive] \\
  C_1[+nareal] &\rightarrow C_1[+fr, +v] \\
  C_1[+pa] C_2[+av] &\rightarrow C_2 C_1
\end{alignat*}

\section{Phoneme inventory}

Thus the following phonemes are present in \lname:

\begin{table}[h]
  \caption{The consonants of \lname.}
  \centering
  \begin{tabular}{l|lllllll}
      & Bilabial & Dental & Alveolar & Palatal & Velar & Uvular & Pharyng. \\
      \hline
      Nasal & m & & n & ɲ & ŋ & & \invalid \\
      Plosive & p b & & t d & c ɟ & k ɡ & & \\
      & pˠ bˠ & & tˤ dˤ & cˤ ɟˤ & & & \\
      Fricative & f v & θ ð & s z & ʃ ʒ & x ɣ & χ ʁ & ħ ʕ \\
      & fˠ vˠ & θˤ ðˤ & sˤ zˤ & ʃˤ ʒˤ & & & \\
      (coärt'd) & fx vɣ & θx ðɣ & θ̼ ð̼ & fʃ vʒ & & \invalid & \invalid \\
      & & & & fʃˠ vʒˠ & & \invalid & \invalid \\
      (whistled) & \invalid & \invalid & s͎ z͎ & ʃ͎ ʒ͎ & \invalid & \invalid & \invalid \\ 
      Affricate & & & ts & tʃ & & & \\
      Lat. fricative & \invalid & & ɬ ɮ & & & & \invalid \\
      Approximant & & & ɹ & & & & \\
      Lat. approx. & \invalid & & l & & ɫ & & \invalid \\
      Tap & & & ɾ & & \invalid & \invalid & \invalid \\
      Trill & & & r & & \invalid & & \invalid \\
      Click & ʘ & & ǀ & ǂ & \invalid & \invalid & \invalid \\
  \end{tabular}
\end{table}

\begin{table}[h]
  \centering
    \caption{The vowels of \lname.}
    \begin{tabular}{l|lll}
        & Front & Central & Back \\
        \hline
        High & i & & u \\
        Mid & e & & o \\
        Low & & a & \\
    \end{tabular}
\end{table}

In addition to consonants and vowels, \lname{} has rod signals, represented by numbers. Rod A is blue and held by one's dominant hand and B is red and held by one's non-dominant hand. Rod signals can occur only at the end of words.

\begin{enumerate}
    \item Rod A is raised to one's chest, while B is pointed down.
    \item Rods A and B are crossed in the front.
    \item Rod B is raised upwards in front of the nondominant arm, while rod A is lowered.
    \item Rod A is pointed sideways near one's nondominant arm, while rod B is lowered.
    \item Rods A and B are extended to the sides.
    \item Rods A and B are extended, facing forward.
    \item Rod A is raised forward, while B is pointed to the side.
    \item Rod B is raised forward, while A is pointed to the side.
    \item Rod A is raised besides one's head, while Rod B is extended toward the side of the dominant hand. This rod signal does not exist alone, but rather as a transition to the seventh or eighth rod signal.
\end{enumerate}

In addition, the fourth rod signal has a ``halfway'' form where Rod A is retracted away from the nondominant arm.

Lowering both rods is interpreted as an absence of a rod signal.

If the use of rods are unavailable, the numerals of the positions may be pronounced.

\section{Hacmisation}

These are hacmised as such:

\begin{table}[h!]
  \caption{The consonants of \lname.}
  \centering
  \begin{tabular}{l|>{\kardinal}l>{\kardinal}l>{\kardinal}l>{\kardinal}l>{\kardinal}l>{\kardinal}l>{\kardinal}l}
      & \textnormal{Bilabial} & \textnormal{Dental} & \textnormal{Alveolar} & \textnormal{Palatal} & \textnormal{Velar} & \textnormal{Uvular} & \textnormal{Pharyng.} \\
      \hline
      Nasal & m & & n & n\^y & n\^g & & \invalid \\
      Plosive & p b & & t d & t\^y d\^y & k g & & \\
      & p\,g b\,g & & t\,g d\,g & t\^y\,g d\^y\,g & & & \\
      Fricative & f v & s\^f z\^v & s z & x j & k\^x g\^j & k\^. g\^. & h h\^j \\
      & f\,g v\,g & s\^f\,g z\^v\,g & s\,g z\,g & x\,g j\,g & & & \\
      (coärt'd) & f\^h v\^h & s\^h z\^h & f\^s v\^z & f\^x v\^z & & \invalid & \invalid \\
      & & & & f\^x\,g v\^j\,g & & \invalid & \invalid \\
      (whistled) & \invalid & \invalid & s\^w z\^w & x\^w j\^w & \invalid & \invalid & \invalid \\ 
      Affricate & & & t\^s & t\^x & & & \\
      Lat. fricative & \invalid & & x\^l j\^l & & & & \invalid \\
      Approximant & & & r & & & & \\
      Lat. approx. & \invalid & & l & & l\,g & & \invalid \\
      Tap & & & c & & \invalid & \invalid & \invalid \\
      Trill & & & c\^c & & \invalid & & \invalid \\
      Click & p\^k & & t\^k & k\^k & \invalid & \invalid & \invalid \\
  \end{tabular}
\end{table}

\begin{table}[h]
  \centering
    \caption{The vowels of \lname.}
    \begin{tabular}{l|>{\kardinal}l>{\kardinal}l>{\kardinal}l}
        & \textnormal{Front} & \textnormal{Central} & \textnormal{Back} \\
        \hline
        High & i & & u \\
        Mid & e & & o \\
        Low & & a & \\
    \end{tabular}
\end{table}

\section{Neðam}

As with its predecessor, \lname{} uses the \emph{Neðam} (\emph{Nsðm} / \textkardinal{nsz\^vm} / \textnedham{neðam}) script. However, the orthography reflects Middle Rymakonian spelling, so it is quite deep. For instance, \hortho{nsz\^vm} /nsðm/ \emph{rose} is written \ortho{\textnedham{neðam}}, reflecting MR \hortho{nez\^vam} /neðam/. The dictionary provides both a hacm and a Neðam spelling for each entry.

\section{Phonotactics}

There seem to be very few restrictions on which phonemes can border each other. However, there is no known case of two adjacent vowels.

Consonant clusters that are difficult to pronounce can be broken up with epenthetic vowels.

\begin{itemize}
  \item /θ̼ ð̼/ are always epenthetised -- epenthetic vowels are inserted before and after occurrences thereof, except at word boundaries.
  \item An epenthetic vowel is inserted between two adjacent sibilants or between an affricate and a sibilant, even between word boundaries.
\end{itemize}

\chapter{Syntax}

\section{Basic word order}

The basic word order is (X)VSO. Descriptors follow what they modify.

Unlike in Middle Rymakonian, there is no special treatment of historically clausal arguments. For instance, if S was a conjunctional phrase, then it would precede V in Middle Rymakonian but follow V in Modern Rymakonian.

Quite strangely, \lname{} uses postpositions.

\section{Questions}

In all questions, the intonation of the second word of the last clause is lowered considerably.

Binary questions have the interrogative polarity marker and no change to syntax.

In wh-questions, the wh-word is pulled to the front (i.~e.~before the verb). This requires case marking for the wh-word: \\
~\\
{}[TBD updated example for 7\_1\_1]

\section{Multiple clauses}

A sentence might have multiple clauses. Each clause in a sentence follows the basic VSO order, and clauses are separated with commas.

\subsection{Relative clauses}

Relativisation is done using the non-reduction strategy. The relative clause is followed by one of the following particles:

\begin{itemize}
  \item \hortho{r} / \northo{ra} if the subject is the one participating in the main clause
  \item \hortho{rg\^j} / \northo{ro} if the direct object is participating
  \item \hortho{rkn} / \northo{rkn} if the object of the postposition \hortho{kn} is participating
\end{itemize}

\chapter{Nouns}

Nouns are declined for number and case.

\section{Number}

Countable nouns come in two numbers: \emph{dual} and \emph{non-dual}.

There are two different conceptualisations of the dual number. Some dialects use the dual number to refer to all cases with two objects (we say that they have the \emph{unpaired dual}); others use it only to refer to objects in pairs (these lack the unpaired dual). In general, dialects without the unpaired dual are more prevalent in cities, as well as northern regions.

Each countable noun has \emph{an inherent number}. A noun whose number agrees with its inherent number receives no marking; a mismatch causes the noun to receive a special affix.

\section{Case}

In a clause with both the subject and object directly expressed in that order, both the subject and object are declined in the nominative case (and their roles are inferred through word order). In a clause where only one is present, or where both are expressed in the opposite order, the subject will receive the nominative case and the object will receive the accusative case.

\section{Noun classes}

There are three overarching groups of noun classes.

\begin{enumerate}
  \item Countable
  \begin{enumerate}
    \item Sentient -- such as humans, AIs, deities.
    \item Non-sentient -- anything else.
  \end{enumerate}
  \item Measurable
  \begin{enumerate}
    \setcounter{enumi}{2}
    \item Measure -- all measurable nouns, especially units of measurement.
  \end{enumerate}
  \item Uncountable
  \begin{enumerate}
    \setcounter{enumi}{3}
    \item Edible -- edible (to humans).
    \item Inedible -- inedible (to humans).
    \item Abstract -- abstract ideas.
  \end{enumerate}
  \item Regular
  \begin{enumerate}
    \setcounter{enumi}{6}
    \item Regular -- these nouns -- primarily borrowed words -- are regularly declined.
  \end{enumerate}
\end{enumerate}

\section{Declensions}

All nouns that are not in the regular class are (or appear to be) declined irregularly in terms of surface declensions; therefore, the dictionary lists all of the declensions. However, they show more regularity when written in Neðam, depending only on the noun class and the final consonant glyph.

Declined forms of nouns are written according to their direct nominative forms, taking care of most of the irregularities. However, \ortho{\textnedham{k t}} are written \ortho{\textnedham{ť č}} before \ortho{\textnedham{i}}.

In addition, Neðam orthography reflects Middle Rymakonian vowel harmony: \northo{e i} represent front vowels and \northo{o u} represent back vowels. \northo{a y} are neutral. Affixes thus match vowel harmony with the root, with \northo{e} corresponding to \northo{o} and \northo{i} with \northo{u}.

\subsection{Countable classes}

\newcommand{\overcol}{& \textnormal{Direct \#} & \textnormal{Inverse \#}}
\begin{longtabu}{|r|>{\nedham}Y|>{\nedham}Y|}
    \caption{Declensions for countable nouns. \label{table:ndecc}} \\
    
    \hline
    \overcol \\
    \endfirsthead
    
    \hline
    \overcol \\
    \hline
    \endhead
    
    \hline
    \endfoot
    
    \hline
    \endlastfoot
    
    \hline
    \multicolumn{3}{|l|}{\textnormal{Sentient: \hortho{nz} / \northo{maza} ``person''}} \\
    \hline
    Nominative & maza & maza\hliii{l} \\
    Accusative & maza\hliii{n} & maza\hliii{nal} \\
    \hline
    \multicolumn{3}{|l|}{\textnormal{Sentient: \hortho{s\^fsn} / \northo{þ.aen} ``magician''}} \\
    \hline
    Nominative & þ.aen & þ.ae\hliii{l} \\
    Accusative & þ.ae\hliii{zin} & þ.ae\hliii{ril} \\
    \hline
    \multicolumn{3}{|p{\linewidth}|}{(Note that the final consonant is preserved only in the direct nominative form.)} \\
    \hline
    \multicolumn{3}{|l|}{\textnormal{Non-sentient: \hortho{mg\^j} / \northo{myŋo} ``rabbit''}} \\
    \hline
    Nominative & myŋo & myŋo\hliii{.u} \\
    Accusative & myŋo\hliii{m} & myŋo\hliii{vu} \\
    \hline
    \multicolumn{3}{|l|}{\textnormal{Non-sentient: \hortho{xnsn} / \northo{.imen} ``house''}} \\
    \hline
    Nominative & .ime\hliii{n} & .ime\hliii{.i} \\
    Accusative & .ime\hliii{zim} & .ime\hliii{rivi} \\
\end{longtabu}

\subsection{Measurable and uncountable classes}

\newcommand{\overcolm}{& \textnormal{Direct}}
\begin{longtabu}{|r|>{\nedham}Y|}
    \caption{Declensions for measurable and uncountable nouns. \label{table:ndecm}} \\
    
    \hline
    \overcolm \\
    \endfirsthead
    
    \hline
    \overcolm \\
    \hline
    \endhead
    
    \hline
    \endfoot
    
    \hline
    \endlastfoot
    
    \hline
    \multicolumn{2}{|l|}{\textnormal{Measure: \hortho{rsm} / \northo{rymy} ``day (continuous)''}} \\
    \hline
    Nominative & rymy \\
    Accusative & rymy\hliii{n} \\
    \hline
    \multicolumn{2}{|l|}{\textnormal{Measure: \hortho{ml} / \northo{mel} ``volume'' (in expressions such as \hortho{*ml-rsg\^j} ``cupful'')}} \\
    \hline
    Nominative & me\hliii{l} \\
    Accusative & me\hliii{zin} \\
    \hline
    \multicolumn{2}{|l|}{\textnormal{Edible: \hortho{tsr} / \northo{teri.} ``beef''}} \\
    \hline
    Nominative & teri. \\
    Accusative & teri.\hliii{n} \\
    \hline
    \multicolumn{2}{|l|}{\textnormal{Edible: \hortho{mn} / \northo{man} ``rice''}} \\
    \hline
    Nominative & ma\hliii{n} \\
    Accusative & ma\hliii{nin} \\
    \hline
    \multicolumn{2}{|l|}{\textnormal{Inedible: \hortho{rg\^jt} / \northo{ruot} ``gold''}} \\
    \hline
    Nominative & ruot \\
    Accusative & ruot\hliii{be} \\
    \hline
    \multicolumn{2}{|l|}{\textnormal{Inedible: \hortho{kcs} / \northo{kařas} ``stone''}} \\
    \hline
    Nominative & kařas \\
    Accusative & kařas\hliii{pe} \\
    \hline
    \multicolumn{2}{|l|}{\textnormal{Abstract: \hortho{fsv} / \northo{fhumo} ``empathy''}} \\
    \hline
    Nominative & fhumo \\
    Accusative & fhumo\hliii{ŋ} \\
    \hline
    \multicolumn{2}{|l|}{\textnormal{Abstract: \hortho{gsx} / \northo{gis} ``[the number] five''}} \\
    \hline
    Nominative & gis \\
    Accusative & gi\hliv{z}\hliii{iŋ} \\
    \hline
    \multicolumn{2}{|c|}{Here, the final consonant is voiced if it is a fricative.} \\
\end{longtabu}

The regular class is declined regularly:

\begin{table}[h]
  \caption{Declensions in the regular class.}
  \centering
  \begin{tabular}{l|ll}
    & Direct & Inverse \\
    \hline
    Nominative & \hortho{-} / \northo{-} & \hortho{-a} / \northo{-aa} \\
    Accusative & \hortho{-n} / \northo{-n} & \hortho{-na} / \northo{-naa} \\
  \end{tabular}
\end{table}

\section{Pronouns}

Personal pronouns are not divided into first, second and third persons as in most languages. Instead, they fall into six categories that exhibit different behaviour depending on whether they occur as the first non-oblique noun in the clause or elsewhere (second noun, verb inflection, oblique):

\begin{table}[h]
    \caption{Pronoun persons and their functions.}
    \centering
    \begin{tabu} to \textwidth {|l|Y|Y|}
        \hline
        Person & Role in first position & Role elsewhere \\
        \hline
        Near & The speaker. & The first non-oblique argument of the clause. \\
        Far & The listener. & The person with which the first argument is conversing. \\
        Other & A third entity. & An entity that is neither the speaker, the listener nor the first argument. \\
        \hline
        Generic & \multicolumn{2}{l|}{A generic entity (akin to ``one'').}  \\
        \hline
        % no X in multicolumn
        Anaphoric Subject & \multicolumn{2}{p{\dimexpr 2\tabucolX+2\tabcolsep+\arrayrulewidth\relax}|}{The subject of the previous clause. Also used on the verb when an oblique or conjunction is present.} \\
        Anaphoric Object & \multicolumn{2}{l|}{The object of the previous clause.} \\
        \hline
    \end{tabu}
\end{table}

In wh-questions, the wh-word assumes the second position and the other argument becomes the first.

If a clause has no explicit arguments, the first argument is understood to be the subject.

\begin{table}[h]
  \caption{Personal pronouns.}
  \centering
  \begin{tabular}{l|>{\kardinal}l>{\kardinal}l|>{\kardinal}l>{\kardinal}l}
      & \multicolumn{2}{c|}{Nominative} & \multicolumn{2}{c}{Accusative} \\
      \hline
      & \textnormal{Non-dual} & \textnormal{Dual} & \textnormal{Non-dual} & \textnormal{Dual} \\
      \hline
      Near & ta \textnedham{taa} & fzx \textnedham{fizi} & tn \textnedham{tan} & fzxn \textnedham{fizen} \\
      Far & po \textnedham{poo} & bra \textnedham{braa} & pn \textnedham{pon} & brn \textnedham{bran} \\
      Other & ni \textnedham{nii} & kz \textnedham{kazi} & nin \textnedham{niin} & kzn \textnedham{kazen} \\
      Anaph. Sub. & ra \textnedham{ra} & n\^yxr \textnedham{ñiri} & rn \textnedham{ran} & n\^yxrsn \textnedham{ñirin} \\
      Anaph. Obj. & ro \textnedham{ro} & n\^yrg\^j \textnedham{ñuro} & rg\^jn \textnedham{ron} & n\^yrg\^jn \textnedham{ñuron} \\
      \hline
      Generic & \multicolumn{2}{>{\kardinal}c|}{u \textnedham{.uu}} & \multicolumn{2}{>{\kardinal}c}{un \textnedham{.uun}} \\
  \end{tabular}
\end{table}

\section{Compounding}

Nouns can be compounded together in a head-initial manner. When that happens, only the leftmost noun is the one to be declined.

{}[TODO example]

\section{Possession}

``X's Y'' is translated as \hortho{\textnormal{Y=}sm \textnormal{X}} (plus phonorun reduction). The clitic is spelt \northo{ma} in the Neðam script. The possessive construction is also used to create appositives. (Note the head-marking!)

{}[TODO example]

\appendix

\chapter{Dictionary}

An entry looks like this:

\textkardinal{mg\^j} \textit{nnonsent}
\quad \textnedham{myŋo} \quad Inflections: \hortho{mg\^j n\^ggm n\^ggv n\^ggv} \quad < MR \hortho{mqn\^go} < LT \hortho{mun\^go} \quad rabbit

From left to right:

\begin{enumerate}
    \item The entry -- the \lname{} term listed.
    \item The part of speech of the corresponding entry:
    \begin{itemize}
        \item \textit{n} -- a noun
        \begin{itemize}
          \item \textit{-d-} -- inherently dual
          \item \textit{-sent} -- sentient noun
          \item \textit{-nonsent} -- nonsentient noun
          \item \textit{-meas} -- measure noun
          \item \textit{-edib} -- edible noun
          \item \textit{-ined} -- inedible noun
          \item \textit{-abst} -- abstract noun
        \end{itemize}
        \item \textit{v1}, \textit{v2}, \textit{v3} -- first-, second- and third- conjugation verbs
        \item \textit{desc} -- a descriptor
        \item \textit{pp} -- a preposition
        \item \textit{-(b)} -- this entry has only neutral vowels when written in Neðam but acts as if it had back vowels
        \item \textit{-(ŋ)} -- certain prefixes will revert the initial \textnedham{ñ} to \textnedham{ŋ}
    \end{itemize}
    \item The spelling in the Neðam script.
    \item The inflections for this word.
    \begin{itemize}
      \item For countable nouns, the order is (direct nominative) → (direct accusative) → (inverse nominative) → (inverse accusative).
      \item For uncountable nouns, the order is (nominative) → (accusative).
    \end{itemize}
    \item The etymology of this word.
    \begin{itemize}
      \item MR stands for Middle Rymakonian.
      \item LT stands for Lek-Tsaro.
    \end{itemize}
    \item The definition -- the gloss for the corresponding entry.
    \begin{itemize}
        \item (S) -- subject
        \item (O) -- direct object
    \end{itemize}
    \item If applicable, any special grammatical or semantic notes for this term.
    \item Optionally, examples of usage.
\end{enumerate}

\begin{multicols}{2}
    \input{7_1_1/dict/dict.tex}
\end{multicols}

\end{document}