\documentclass{book}

\usepackage[shortsuper,hacm,ltfont,nedham]{common/uruwi}

\newcommand{\lname}{Modern Rymakonian}

\title{lek-\bs{}rqmako-ma lek-ma st\^xee-\bs{}rqmako}
\author{uruwi}

\begin{document}

\pagecolor{Thistle!25}

\begin{titlepage}
    \makeatletter
    \begin{center}
        {\color{Orchid} \hprule \vspace{1.5ex} \\}
        %{\Huge \ltfont \textcolor{Plum}{LKe\bs{}TSxaRMoa SL LKeMa SL STfXe\bs{}RMyaKo}\\}
        {\Huge \kardinal \textcolor{Purple}{\@title} \\}
        {\large \textit{\lname}, the language of \textit{Rymako} \\}
        {\color{Orchid} \hprule \vspace{1.5ex} \\}
        % ----------------------------------------------------------------
        \vspace{1.5cm}
        {\Large\bfseries \@author}\\[5pt]
        %uruwi@protonmail.com\\[14pt]
        % ----------------------------------------------------------------
        \vspace{2cm}
        %{\Large\textlt{IVvoQMxeBPieLBxf TXxeKy}} \\
        \textkardinal{een\^gs.-meibpelbe-kona} \\[5pt]
        \emph{A complete grammar}\\[2cm]
        %{in partial fulfilment for the award of the degree of} \\[2cm]
        %\tsc{\Large{{Doctor of Philosophy}}} \\[5pt]
        %{in some subject} \vspace{0.4cm} \\[2cm]
        % {By}\\[5pt] {\Large \sc {Me}}
        \vfill
        % ----------------------------------------------------------------
        %\includegraphics[width=0.19\textwidth]{example-image-a}\\[5pt]
        %{blah}\\[5pt]
        %{blahblah}\\[5pt]
        %{blahblah}\\
        \vfill
        {\@date}
    \end{center}
    \makeatother
\end{titlepage}

\pagecolor{Thistle!15}

\begin{center}
    \textit{Dedicated to Gufferdk.}
\end{center}

\begin{verbatim}
Branch: canon
Version: 0.1
Date: 2018-04-30 (29 dyu mik)
\end{verbatim}

(C)opyright 2018 Uruwi. See README.md for details.

\tableofcontents

\section{Introduction}

\chapter{Phonology and orthography}

\section{Diachronics}

\subsection{L-recession: Kasnepy 430 -- 490}

The first set of sound changes involves the loss of /l/.

\begin{alignat*}{2}
  % \alpha &\rightarrow \omega &\quad(\lambda \blacklozenge \rho) &\quad[\Gamma]
  \text{ʔ} &\rightarrow \emptyset \\
  C_1[+ap] \text{l} &\rightarrow C_1 \\
  C_1[+na] \text{l} &\rightarrow \text{ŋ} \\
  \text{ɾl} &\rightarrow \text{r} \\
  C_1[+lb] \text{l} &\rightarrow C_1[+velarised] \\
  C_1[+ve] \text{l} &\rightarrow C_1 \\
  C_1[+lf] \text{l} &\rightarrow C_1 \\
  C_1[+whistled] \text{l} &\rightarrow C_1[+lf, -whistled] \\
  C_1[+affricate] \text{l} &\rightarrow C_1 \\
  C_1 \text{l} &\rightarrow C_1[+pharyngealised] \\
  V_1 \text{l} &\rightarrow V_1[+lateralised] \\
  \text{l} &\rightarrow \text{dˤ}  &\quad(\square \blacklozenge) \\
  \{\text{ʌ}, \text{ɯ}\} &\rightarrow \{\text{o}, \text{u}\} &\quad \lnot(\text{w} \blacklozenge) \\
  \text{ɯ} &\rightarrow \text{i}
\end{alignat*}

(The observant reader might notice the short timespan of these changes. This is not an error.)

\subsection{Vocaloëxodus: Kasnepy 660 -- Nihel 50}

At this point, vowels start to be lost. The first one to be lost is /ʉ̜/:

\begin{alignat*}{2}
  % \alpha &\rightarrow \omega &\quad(\lambda \blacklozenge \rho) &\quad[\Gamma]
  \text{ʉ̜} &\rightarrow \emptyset &\quad(\blacklozenge \square \lor \square \blacklozenge) \\
  \text{ʉ̜} &\rightarrow \text{u} &\quad(C_1[+lb] \blacklozenge) \\
  \text{wu} &\rightarrow \text{u} \\
  \text{jʉ̜} &\rightarrow \text{i} \\
  \text{ʉ̜} &\rightarrow \text{o} &\quad(C_1[+ve] \blacklozenge) \\
  \text{ʉ̜} &\rightarrow \text{e} &\quad(C_1 \blacklozenge) \\
  \text{ʉ̜} &\rightarrow \emptyset \\
  \text{ʉ̜ˡ} &\rightarrow \text{ɹ}
\end{alignat*}

This is followed by vowel merging:

\begin{alignat*}{2}
  % \alpha &\rightarrow \omega &\quad(\lambda \blacklozenge \rho) &\quad[\Gamma]
  V_1V_1 &\rightarrow V_1[+l] \\
  \text{i} V_1 &\rightarrow \text{j} V_1 \\
  \text{u} V_1 &\rightarrow \text{w} V_1
\end{alignat*}

After this change, \emph{lateral rotation} takes place: lateralisation transfers from one vowel to the next within a word, wrapping back to the first vowel from the last. Thus, /toˡu/ becomes /touˡ/ -- the lateralisation transfers from the first vowel to the second.

(Short) /u/ is the next vowel to be lost:

\begin{alignat*}{2}
  % \alpha &\rightarrow \omega &\quad(\lambda \blacklozenge \rho) &\quad[\Gamma]
  \{\text{u}, \text{uˡ}\} &\rightarrow \emptyset &\quad(\square \blacklozenge) \\
  \{\text{u}, \text{uˡ}\} &\rightarrow \text{v} &\quad(V_1 \blacklozenge) \\
  C_1\{\text{k}, \text{ɡ}, \text{x}, \text{ɣ}, \text{ŋ}\} V_1\{\text{u}, \text{uˡ}\} &\rightarrow
    C_1\{\text{p}, \text{b}, \text{f}, \text{v}, \text{m}\} V_1\{\text{a}, \text{aˡ}\} \\
  C_1\{\text{fx}, \text{vɣ}, \text{θx}, \text{ðɣ}\} V_1\{\text{u}, \text{uˡ}\} &\rightarrow
    C_1\{\text{f}, \text{v}, \text{θ̼}, \text{ð̼}\} V_1\{\text{e}, \text{eˡ}\} \\
  C_1\{\text{s}, \text{z}, \text{ʃ}, \text{ʒ}\} \text{u} &\rightarrow
    C_1\{\text{s͎}, \text{z͎}, \text{ʃ͎}, \text{ʒ͎}\} \text{e} \\
  C_1\{\text{s}, \text{z}, \text{ʃ}, \text{ʒ}\} \text{uˡ} &\rightarrow
    C_1\{\text{ɬ}, \text{ɮ}, \text{ɬ}, \text{ɮ}\} \text{e} \\
  \text{u} &\rightarrow \emptyset \\
  \text{uˡ} &\rightarrow \text{ɹ} \\
  \text{w} &\rightarrow \text{v}
\end{alignat*}

After /u/, /i e ʌ/ (and their lateral counterparts) are lost:

\begin{alignat*}{2}
  % \alpha &\rightarrow \omega &\quad(\lambda \blacklozenge \rho) &\quad[\Gamma]
  \{\text{i}, \text{e}, \text{ʌ}\} &\rightarrow
    \emptyset &\quad(\blacklozenge \square) \\
  \{\text{iˡ}, \text{eˡ}, \text{ʌˡ}\} &\rightarrow
    \{\text{l}, \text{l}, \text{ɫ}\} \\
  \{\text{i}, \text{e}, \text{ʌ}\} C_1 &\rightarrow
    \emptyset &\quad(C_1 \blacklozenge) &\quad[\#\delta > 3] \\
  \text{e} &\rightarrow \emptyset &\quad(C_1[+whistled] \blacklozenge) \\
  \{\text{i}, \text{e}, \text{ʌ}\} &\rightarrow
    \{\text{ç}, \text{s}, \text{θx}\}
\end{alignat*}

/o/ is the next vowel to be lost:

\begin{alignat*}{2}
  % \alpha &\rightarrow \omega &\quad(\lambda \blacklozenge \rho) &\quad[\Gamma]
  \text{oˡ} &\rightarrow \text{ɫ} \\
  C_1[+na] \text{o} &\rightarrow C_1[+nareal] \\
  \{\text{p}, \text{t}, \text{c}\} \text{o} &\rightarrow
    \{\text{ʘ}, \text{ǀ}, \text{ǂ}\} \\
  \{\text{b}, \text{d}, \text{ɟ}, \text{ɡ}\} \text{o} &\rightarrow
    \{\text{p}, \text{t}, \text{c}, \text{k}\} \\
  \{\text{m}, \text{n}, \text{ɲ}, \text{ŋ}\} \text{o} &\rightarrow
    \{\text{b}, \text{d}, \text{ɟ}, \text{ɡ}\} \\
  \{\text{f}, \text{v}\} \text{o} &\rightarrow \text{p} &\quad(\blacklozenge \{\square, C_2[+ap], C_2[+la]\}) \\
  \{\text{θ}, \text{ð}, \text{s}, \text{z}, \text{s͎}, \text{z͎}\} \text{o} &\rightarrow \text{t} &\quad(\ldots) \\
  \{\text{ç}, \text{ʃ}, \text{ʒ}, \text{ʃ͎}, \text{ʒ͎}, \text{x}, \text{ɣ}\} \text{o} &\rightarrow \text{k} &\quad(\ldots) \\
  \{\text{f}, \text{v}, \text{θ}, \text{ð}\} \text{o} &\rightarrow
    \{\text{ʕ}, \text{ħ}, \text{ʁ}, \text{χ}\} \\
  \text{o} &\rightarrow C_1[+fr] &\quad(\square \blacklozenge C_1[+pl]) \\
  \text{o} &\rightarrow \text{ɣ}
\end{alignat*}

Finally /a/ is lost: $\{\text{a}, \text{aˡ}\} \rightarrow \emptyset$. The long vowels can subsequently be reänalysed as being short.

It should be noted that epenthetic vowels are allowed between consonants.

\subsection{Cluster reduction: Nihel 70 -- 130}

The consonant clusters resulting from the previous vocaloëxodus turn out to be quite complex. Let $f$ be as such:

\begin{align*}
  f(p) &=
  \begin{cases}
    3 & p \text{ is voiced or pharyngealised} \\
    2 & p = \text{k} \text{ or $p$ is velarised} \\
    1 & p = \text{t} \\
    0 & p \in \{\text{p}, \text{c}\}
  \end{cases}
\end{align*}

Then

\begin{alignat*}{2}
  % \alpha &\rightarrow \omega &\quad(\lambda \blacklozenge \rho) &\quad[\Gamma]
  C_1[+pl] C_2[+pl] &\rightarrow C_1 &\quad \lnot(V_1 \blacklozenge V_2) &\quad[f(C_1) \ge f(C_2)] \\
  C_1[+pl] C_2[+pl] &\rightarrow C_2 &\quad \lnot(V_1 \blacklozenge V_2) &\quad[f(C_2) > f(C_1)] \\
  C_1\{\text{l}, \text{ɫ}, \text{ɹ}\} C_2[+pl] &\rightarrow C_2 C_1 &\quad(C_3 \blacklozenge) \\
  C_1[+na] &\rightarrow C_1[pa=x] &\quad(C_2[-na,pa=x]) \\
  C_1[+click] C_2[+pl, -ve, -v] &\rightarrow C_2[+click] \\
  C_1[+pl, -ve, -v] C_2[+click] &\rightarrow C_1[+click] \\
  C_1[+click] C_2[+pl, -ve, +v] &\rightarrow C_2[+implosive] \\
  C_1[+pl, -ve, +v] C_2[+click] &\rightarrow C_1[+implosive] \\
  C_1[+nareal] &\rightarrow C_1[+fr, +v]
\end{alignat*}

\section{Phoneme inventory}

Thus the following phonemes are present in \lname:

\begin{table}[h]
  \caption{The consonants of \lname.}
  \centering
  \begin{tabular}{l|lllllll}
      & Bilabial & Dental & Alveolar & Palatal & Velar & Uvular & Pharyng. \\
      \hline
      Nasal & m & & n & ɲ & ŋ & & \invalid \\
      Plosive & p b & & t d & c ɟ & k ɡ & & \\
      & pˠ bˠ & & tˤ dˤ & cˤ ɟˤ & & & \\
      Fricative & f v & θ ð & s z & ʃ ʒ & x ɣ & χ ʁ & ħ ʕ \\
      & fˠ vˠ & θˤ ðˤ & sˤ zˤ & ʃˤ ʒˤ & & & \\
      (coärt'd) & fx vɣ & θx ðɣ & θ̼ ð̼ & fʃ vʒ & & \invalid & \invalid \\
      & & & & fʃˠ vʒˠ & & \invalid & \invalid \\
      (whistled) & \invalid & \invalid & s͎ z͎ & ʃ͎ ʒ͎ & \invalid & \invalid & \invalid \\ 
      Affricate & & & ts & tʃ & & & \\
      Lat. fricative & \invalid & & ɬ ɮ & & & & \invalid \\
      Approximant & & & ɹ & & & & \\
      Lat. approx. & \invalid & & l & & ɫ & & \invalid \\
      Tap & & & ɾ & & \invalid & \invalid & \invalid \\
      Trill & & & r & & \invalid & & \invalid \\
      Click & ʘ & & ǀ & ǂ & \invalid & \invalid & \invalid \\
  \end{tabular}
\end{table}

\begin{table}[h]
  \centering
    \caption{The vowels of \lname.}
    \begin{tabular}{l|lll}
        & Front & Central & Back \\
        \hline
        High & i & & u \\
        Mid & e & & o \\
        Low & & a & \\
    \end{tabular}
\end{table}

\section{Hacmisation}

These are hacmised as such:

\begin{table}[h!]
  \caption{The consonants of \lname.}
  \centering
  \begin{tabular}{l|>{\kardinal}l>{\kardinal}l>{\kardinal}l>{\kardinal}l>{\kardinal}l>{\kardinal}l>{\kardinal}l}
      & \textnormal{Bilabial} & \textnormal{Dental} & \textnormal{Alveolar} & \textnormal{Palatal} & \textnormal{Velar} & \textnormal{Uvular} & \textnormal{Pharyng.} \\
      \hline
      Nasal & m & & n & n\^y & n\^g & & \invalid \\
      Plosive & p b & & t d & t\^y d\^y & k g & & \\
      & p\,g b\,g & & t\,g d\,g & t\^y\,g d\^y\,g & & & \\
      Fricative & f v & s\^f z\^v & s z & x j & k\^x g\^j & k\^. g\^. & h h\^j \\
      & f\,g v\,g & s\^f\,g z\^v\,g & s\,g z\,g & x\,g j\,g & & & \\
      (coärt'd) & f\^h v\^h & s\^h z\^h & f\^s v\^z & f\^x v\^z & & \invalid & \invalid \\
      & & & & f\^x\,g v\^j\,g & & \invalid & \invalid \\
      (whistled) & \invalid & \invalid & s\^w z\^w & x\^w z\^w & \invalid & \invalid & \invalid \\ 
      Affricate & & & t\^s & t\^x & & & \\
      Lat. fricative & \invalid & & x\^l j\^l & & & & \invalid \\
      Approximant & & & r & & & & \\
      Lat. approx. & \invalid & & l & & l\,g & & \invalid \\
      Tap & & & c & & \invalid & \invalid & \invalid \\
      Trill & & & c\^c & & \invalid & & \invalid \\
      Click & p\^k & & t\^k & k\^k & \invalid & \invalid & \invalid \\
  \end{tabular}
\end{table}

\begin{table}[h]
  \centering
    \caption{The vowels of \lname.}
    \begin{tabular}{l|>{\kardinal}l>{\kardinal}l>{\kardinal}l}
        & \textnormal{Front} & \textnormal{Central} & \textnormal{Back} \\
        \hline
        High & i & & u \\
        Mid & e & & o \\
        Low & & a & \\
    \end{tabular}
\end{table}

\section{Neðam}

As with its predecessor, \lname{} uses the \emph{Neðam} (\emph{Nsðm} / \textkardinal{nsz\^vm} / \textnedham{neðam}) script. However, the orthography reflects Middle Rymakonian spelling, so it is quite deep. For instance, \hortho{nsz\^vm} /nsðm/ \emph{rose} is written \ortho{\textnedham{neðam}}, reflecting MR \hortho{nez\^vam} /neðam/. The dictionary provides both a hacm and a Neðam spelling for each entry.

\chapter{Dictionary}

An entry looks like this:

\textkardinal{mak-} \textit{v1}
\quad (S) eats (O)

From left to right:

\begin{enumerate}
    \item The entry -- the \lname{} term listed.
    \item The part of speech of the corresponding entry:
    \begin{itemize}
        \item \textit{n} -- a noun
        \begin{itemize}
          \item \textit{-d-} -- inherently dual
          \item \textit{-sent} -- sentient noun
          \item \textit{-nonsent} -- nonsentient noun
          \item \textit{-meas} -- measure noun
          \item \textit{-edib} -- edible noun
          \item \textit{-ined} -- inedible noun
          \item \textit{-abst} -- abstract noun
        \end{itemize}
        \item \textit{v1}, \textit{v2}, \textit{v3} -- first-, second- and third- conjugation verbs
        \item \textit{desc} -- a descriptor
        \item \textit{pp} -- a preposition
        \item \textit{-(b)} -- this entry has only neutral vowels but acts as if it had back vowels
        \item \textit{-(ŋ)} -- this entry came from a word that started with \hortho{n\^g-} and thus certain prefixes will revert it back
    \end{itemize}
    \item The definition -- the gloss for the corresponding entry.
    \begin{itemize}
        \item (S) -- subject
        \item (O) -- direct object
    \end{itemize}
    \item If applicable, any special grammatical or semantic notes for this term.
    \item Optionally, examples of usage.
\end{enumerate}

\begin{multicols}{2}
    \input{7_1_1/dict/dict.tex}
\end{multicols}

\end{document}